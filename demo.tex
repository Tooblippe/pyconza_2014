
% Default to the notebook output style

    


% Inherit from the specified cell style.




    
\documentclass{article}

    
    
    \usepackage{graphicx} % Used to insert images
    \usepackage{adjustbox} % Used to constrain images to a maximum size 
    \usepackage{color} % Allow colors to be defined
    \usepackage{enumerate} % Needed for markdown enumerations to work
    \usepackage{geometry} % Used to adjust the document margins
    \usepackage{amsmath} % Equations
    \usepackage{amssymb} % Equations
    \usepackage[mathletters]{ucs} % Extended unicode (utf-8) support
    \usepackage[utf8x]{inputenc} % Allow utf-8 characters in the tex document
    \usepackage{fancyvrb} % verbatim replacement that allows latex
    \usepackage{grffile} % extends the file name processing of package graphics 
                         % to support a larger range 
    % The hyperref package gives us a pdf with properly built
    % internal navigation ('pdf bookmarks' for the table of contents,
    % internal cross-reference links, web links for URLs, etc.)
    \usepackage{hyperref}
    \usepackage{longtable} % longtable support required by pandoc >1.10
    \usepackage{booktabs}  % table support for pandoc > 1.12.2
    

    
    
    \definecolor{orange}{cmyk}{0,0.4,0.8,0.2}
    \definecolor{darkorange}{rgb}{.71,0.21,0.01}
    \definecolor{darkgreen}{rgb}{.12,.54,.11}
    \definecolor{myteal}{rgb}{.26, .44, .56}
    \definecolor{gray}{gray}{0.45}
    \definecolor{lightgray}{gray}{.95}
    \definecolor{mediumgray}{gray}{.8}
    \definecolor{inputbackground}{rgb}{.95, .95, .85}
    \definecolor{outputbackground}{rgb}{.95, .95, .95}
    \definecolor{traceback}{rgb}{1, .95, .95}
    % ansi colors
    \definecolor{red}{rgb}{.6,0,0}
    \definecolor{green}{rgb}{0,.65,0}
    \definecolor{brown}{rgb}{0.6,0.6,0}
    \definecolor{blue}{rgb}{0,.145,.698}
    \definecolor{purple}{rgb}{.698,.145,.698}
    \definecolor{cyan}{rgb}{0,.698,.698}
    \definecolor{lightgray}{gray}{0.5}
    
    % bright ansi colors
    \definecolor{darkgray}{gray}{0.25}
    \definecolor{lightred}{rgb}{1.0,0.39,0.28}
    \definecolor{lightgreen}{rgb}{0.48,0.99,0.0}
    \definecolor{lightblue}{rgb}{0.53,0.81,0.92}
    \definecolor{lightpurple}{rgb}{0.87,0.63,0.87}
    \definecolor{lightcyan}{rgb}{0.5,1.0,0.83}
    
    % commands and environments needed by pandoc snippets
    % extracted from the output of `pandoc -s`
    \DefineVerbatimEnvironment{Highlighting}{Verbatim}{commandchars=\\\{\}}
    % Add ',fontsize=\small' for more characters per line
    \newenvironment{Shaded}{}{}
    \newcommand{\KeywordTok}[1]{\textcolor[rgb]{0.00,0.44,0.13}{\textbf{{#1}}}}
    \newcommand{\DataTypeTok}[1]{\textcolor[rgb]{0.56,0.13,0.00}{{#1}}}
    \newcommand{\DecValTok}[1]{\textcolor[rgb]{0.25,0.63,0.44}{{#1}}}
    \newcommand{\BaseNTok}[1]{\textcolor[rgb]{0.25,0.63,0.44}{{#1}}}
    \newcommand{\FloatTok}[1]{\textcolor[rgb]{0.25,0.63,0.44}{{#1}}}
    \newcommand{\CharTok}[1]{\textcolor[rgb]{0.25,0.44,0.63}{{#1}}}
    \newcommand{\StringTok}[1]{\textcolor[rgb]{0.25,0.44,0.63}{{#1}}}
    \newcommand{\CommentTok}[1]{\textcolor[rgb]{0.38,0.63,0.69}{\textit{{#1}}}}
    \newcommand{\OtherTok}[1]{\textcolor[rgb]{0.00,0.44,0.13}{{#1}}}
    \newcommand{\AlertTok}[1]{\textcolor[rgb]{1.00,0.00,0.00}{\textbf{{#1}}}}
    \newcommand{\FunctionTok}[1]{\textcolor[rgb]{0.02,0.16,0.49}{{#1}}}
    \newcommand{\RegionMarkerTok}[1]{{#1}}
    \newcommand{\ErrorTok}[1]{\textcolor[rgb]{1.00,0.00,0.00}{\textbf{{#1}}}}
    \newcommand{\NormalTok}[1]{{#1}}
    
    % Define a nice break command that doesn't care if a line doesn't already
    % exist.
    \def\br{\hspace*{\fill} \\* }
    % Math Jax compatability definitions
    \def\gt{>}
    \def\lt{<}
    % Document parameters
    \title{demo}
    
    
    

    % Pygments definitions
    
\makeatletter
\def\PY@reset{\let\PY@it=\relax \let\PY@bf=\relax%
    \let\PY@ul=\relax \let\PY@tc=\relax%
    \let\PY@bc=\relax \let\PY@ff=\relax}
\def\PY@tok#1{\csname PY@tok@#1\endcsname}
\def\PY@toks#1+{\ifx\relax#1\empty\else%
    \PY@tok{#1}\expandafter\PY@toks\fi}
\def\PY@do#1{\PY@bc{\PY@tc{\PY@ul{%
    \PY@it{\PY@bf{\PY@ff{#1}}}}}}}
\def\PY#1#2{\PY@reset\PY@toks#1+\relax+\PY@do{#2}}

\expandafter\def\csname PY@tok@gd\endcsname{\def\PY@tc##1{\textcolor[rgb]{0.63,0.00,0.00}{##1}}}
\expandafter\def\csname PY@tok@gu\endcsname{\let\PY@bf=\textbf\def\PY@tc##1{\textcolor[rgb]{0.50,0.00,0.50}{##1}}}
\expandafter\def\csname PY@tok@gt\endcsname{\def\PY@tc##1{\textcolor[rgb]{0.00,0.27,0.87}{##1}}}
\expandafter\def\csname PY@tok@gs\endcsname{\let\PY@bf=\textbf}
\expandafter\def\csname PY@tok@gr\endcsname{\def\PY@tc##1{\textcolor[rgb]{1.00,0.00,0.00}{##1}}}
\expandafter\def\csname PY@tok@cm\endcsname{\let\PY@it=\textit\def\PY@tc##1{\textcolor[rgb]{0.25,0.50,0.50}{##1}}}
\expandafter\def\csname PY@tok@vg\endcsname{\def\PY@tc##1{\textcolor[rgb]{0.10,0.09,0.49}{##1}}}
\expandafter\def\csname PY@tok@m\endcsname{\def\PY@tc##1{\textcolor[rgb]{0.40,0.40,0.40}{##1}}}
\expandafter\def\csname PY@tok@mh\endcsname{\def\PY@tc##1{\textcolor[rgb]{0.40,0.40,0.40}{##1}}}
\expandafter\def\csname PY@tok@go\endcsname{\def\PY@tc##1{\textcolor[rgb]{0.53,0.53,0.53}{##1}}}
\expandafter\def\csname PY@tok@ge\endcsname{\let\PY@it=\textit}
\expandafter\def\csname PY@tok@vc\endcsname{\def\PY@tc##1{\textcolor[rgb]{0.10,0.09,0.49}{##1}}}
\expandafter\def\csname PY@tok@il\endcsname{\def\PY@tc##1{\textcolor[rgb]{0.40,0.40,0.40}{##1}}}
\expandafter\def\csname PY@tok@cs\endcsname{\let\PY@it=\textit\def\PY@tc##1{\textcolor[rgb]{0.25,0.50,0.50}{##1}}}
\expandafter\def\csname PY@tok@cp\endcsname{\def\PY@tc##1{\textcolor[rgb]{0.74,0.48,0.00}{##1}}}
\expandafter\def\csname PY@tok@gi\endcsname{\def\PY@tc##1{\textcolor[rgb]{0.00,0.63,0.00}{##1}}}
\expandafter\def\csname PY@tok@gh\endcsname{\let\PY@bf=\textbf\def\PY@tc##1{\textcolor[rgb]{0.00,0.00,0.50}{##1}}}
\expandafter\def\csname PY@tok@ni\endcsname{\let\PY@bf=\textbf\def\PY@tc##1{\textcolor[rgb]{0.60,0.60,0.60}{##1}}}
\expandafter\def\csname PY@tok@nl\endcsname{\def\PY@tc##1{\textcolor[rgb]{0.63,0.63,0.00}{##1}}}
\expandafter\def\csname PY@tok@nn\endcsname{\let\PY@bf=\textbf\def\PY@tc##1{\textcolor[rgb]{0.00,0.00,1.00}{##1}}}
\expandafter\def\csname PY@tok@no\endcsname{\def\PY@tc##1{\textcolor[rgb]{0.53,0.00,0.00}{##1}}}
\expandafter\def\csname PY@tok@na\endcsname{\def\PY@tc##1{\textcolor[rgb]{0.49,0.56,0.16}{##1}}}
\expandafter\def\csname PY@tok@nb\endcsname{\def\PY@tc##1{\textcolor[rgb]{0.00,0.50,0.00}{##1}}}
\expandafter\def\csname PY@tok@nc\endcsname{\let\PY@bf=\textbf\def\PY@tc##1{\textcolor[rgb]{0.00,0.00,1.00}{##1}}}
\expandafter\def\csname PY@tok@nd\endcsname{\def\PY@tc##1{\textcolor[rgb]{0.67,0.13,1.00}{##1}}}
\expandafter\def\csname PY@tok@ne\endcsname{\let\PY@bf=\textbf\def\PY@tc##1{\textcolor[rgb]{0.82,0.25,0.23}{##1}}}
\expandafter\def\csname PY@tok@nf\endcsname{\def\PY@tc##1{\textcolor[rgb]{0.00,0.00,1.00}{##1}}}
\expandafter\def\csname PY@tok@si\endcsname{\let\PY@bf=\textbf\def\PY@tc##1{\textcolor[rgb]{0.73,0.40,0.53}{##1}}}
\expandafter\def\csname PY@tok@s2\endcsname{\def\PY@tc##1{\textcolor[rgb]{0.73,0.13,0.13}{##1}}}
\expandafter\def\csname PY@tok@vi\endcsname{\def\PY@tc##1{\textcolor[rgb]{0.10,0.09,0.49}{##1}}}
\expandafter\def\csname PY@tok@nt\endcsname{\let\PY@bf=\textbf\def\PY@tc##1{\textcolor[rgb]{0.00,0.50,0.00}{##1}}}
\expandafter\def\csname PY@tok@nv\endcsname{\def\PY@tc##1{\textcolor[rgb]{0.10,0.09,0.49}{##1}}}
\expandafter\def\csname PY@tok@s1\endcsname{\def\PY@tc##1{\textcolor[rgb]{0.73,0.13,0.13}{##1}}}
\expandafter\def\csname PY@tok@sh\endcsname{\def\PY@tc##1{\textcolor[rgb]{0.73,0.13,0.13}{##1}}}
\expandafter\def\csname PY@tok@sc\endcsname{\def\PY@tc##1{\textcolor[rgb]{0.73,0.13,0.13}{##1}}}
\expandafter\def\csname PY@tok@sx\endcsname{\def\PY@tc##1{\textcolor[rgb]{0.00,0.50,0.00}{##1}}}
\expandafter\def\csname PY@tok@bp\endcsname{\def\PY@tc##1{\textcolor[rgb]{0.00,0.50,0.00}{##1}}}
\expandafter\def\csname PY@tok@c1\endcsname{\let\PY@it=\textit\def\PY@tc##1{\textcolor[rgb]{0.25,0.50,0.50}{##1}}}
\expandafter\def\csname PY@tok@kc\endcsname{\let\PY@bf=\textbf\def\PY@tc##1{\textcolor[rgb]{0.00,0.50,0.00}{##1}}}
\expandafter\def\csname PY@tok@c\endcsname{\let\PY@it=\textit\def\PY@tc##1{\textcolor[rgb]{0.25,0.50,0.50}{##1}}}
\expandafter\def\csname PY@tok@mf\endcsname{\def\PY@tc##1{\textcolor[rgb]{0.40,0.40,0.40}{##1}}}
\expandafter\def\csname PY@tok@err\endcsname{\def\PY@bc##1{\setlength{\fboxsep}{0pt}\fcolorbox[rgb]{1.00,0.00,0.00}{1,1,1}{\strut ##1}}}
\expandafter\def\csname PY@tok@kd\endcsname{\let\PY@bf=\textbf\def\PY@tc##1{\textcolor[rgb]{0.00,0.50,0.00}{##1}}}
\expandafter\def\csname PY@tok@ss\endcsname{\def\PY@tc##1{\textcolor[rgb]{0.10,0.09,0.49}{##1}}}
\expandafter\def\csname PY@tok@sr\endcsname{\def\PY@tc##1{\textcolor[rgb]{0.73,0.40,0.53}{##1}}}
\expandafter\def\csname PY@tok@mo\endcsname{\def\PY@tc##1{\textcolor[rgb]{0.40,0.40,0.40}{##1}}}
\expandafter\def\csname PY@tok@kn\endcsname{\let\PY@bf=\textbf\def\PY@tc##1{\textcolor[rgb]{0.00,0.50,0.00}{##1}}}
\expandafter\def\csname PY@tok@mi\endcsname{\def\PY@tc##1{\textcolor[rgb]{0.40,0.40,0.40}{##1}}}
\expandafter\def\csname PY@tok@gp\endcsname{\let\PY@bf=\textbf\def\PY@tc##1{\textcolor[rgb]{0.00,0.00,0.50}{##1}}}
\expandafter\def\csname PY@tok@o\endcsname{\def\PY@tc##1{\textcolor[rgb]{0.40,0.40,0.40}{##1}}}
\expandafter\def\csname PY@tok@kr\endcsname{\let\PY@bf=\textbf\def\PY@tc##1{\textcolor[rgb]{0.00,0.50,0.00}{##1}}}
\expandafter\def\csname PY@tok@s\endcsname{\def\PY@tc##1{\textcolor[rgb]{0.73,0.13,0.13}{##1}}}
\expandafter\def\csname PY@tok@kp\endcsname{\def\PY@tc##1{\textcolor[rgb]{0.00,0.50,0.00}{##1}}}
\expandafter\def\csname PY@tok@w\endcsname{\def\PY@tc##1{\textcolor[rgb]{0.73,0.73,0.73}{##1}}}
\expandafter\def\csname PY@tok@kt\endcsname{\def\PY@tc##1{\textcolor[rgb]{0.69,0.00,0.25}{##1}}}
\expandafter\def\csname PY@tok@ow\endcsname{\let\PY@bf=\textbf\def\PY@tc##1{\textcolor[rgb]{0.67,0.13,1.00}{##1}}}
\expandafter\def\csname PY@tok@sb\endcsname{\def\PY@tc##1{\textcolor[rgb]{0.73,0.13,0.13}{##1}}}
\expandafter\def\csname PY@tok@k\endcsname{\let\PY@bf=\textbf\def\PY@tc##1{\textcolor[rgb]{0.00,0.50,0.00}{##1}}}
\expandafter\def\csname PY@tok@se\endcsname{\let\PY@bf=\textbf\def\PY@tc##1{\textcolor[rgb]{0.73,0.40,0.13}{##1}}}
\expandafter\def\csname PY@tok@sd\endcsname{\let\PY@it=\textit\def\PY@tc##1{\textcolor[rgb]{0.73,0.13,0.13}{##1}}}

\def\PYZbs{\char`\\}
\def\PYZus{\char`\_}
\def\PYZob{\char`\{}
\def\PYZcb{\char`\}}
\def\PYZca{\char`\^}
\def\PYZam{\char`\&}
\def\PYZlt{\char`\<}
\def\PYZgt{\char`\>}
\def\PYZsh{\char`\#}
\def\PYZpc{\char`\%}
\def\PYZdl{\char`\$}
\def\PYZhy{\char`\-}
\def\PYZsq{\char`\'}
\def\PYZdq{\char`\"}
\def\PYZti{\char`\~}
% for compatibility with earlier versions
\def\PYZat{@}
\def\PYZlb{[}
\def\PYZrb{]}
\makeatother


    % Exact colors from NB
    \definecolor{incolor}{rgb}{0.0, 0.0, 0.5}
    \definecolor{outcolor}{rgb}{0.545, 0.0, 0.0}



    
    % Prevent overflowing lines due to hard-to-break entities
    \sloppy 
    % Setup hyperref package
    \hypersetup{
      breaklinks=true,  % so long urls are correctly broken across lines
      colorlinks=true,
      urlcolor=blue,
      linkcolor=darkorange,
      citecolor=darkgreen,
      }
    % Slightly bigger margins than the latex defaults
    
    \geometry{verbose,tmargin=1in,bmargin=1in,lmargin=1in,rmargin=1in}
    
    

    \begin{document}
    
    
    \maketitle
    
    

    
    \begin{Verbatim}[commandchars=\\\{\}]
{\color{incolor}In [{\color{incolor}1}]:} \PY{c}{\PYZsh{} Load a custom CSS and import plotting}
        \PY{c}{\PYZsh{} https://github.com/fperez/pycon2014\PYZhy{}keynote}
        
        \PY{o}{\PYZpc{}}\PY{k}{run} \PY{n}{static}\PY{o}{/}\PY{n}{talktools}\PY{o}{/}\PY{n}{talktools}\PY{o}{.}\PY{n}{py} 
        \PY{c}{\PYZsh{}makes sure inline plotting is enabled}
\end{Verbatim}

    
    \begin{verbatim}
<IPython.core.display.HTML at 0x105f12690>
    \end{verbatim}

    
    \begin{Verbatim}[commandchars=\\\{\}]
{\color{incolor}In [{\color{incolor}1}]:} \PY{o}{\PYZpc{}}\PY{k}{pylab} \PY{n}{inline} 
        \PY{c}{\PYZsh{}set figure size}
        \PY{n}{figsize}\PY{p}{(}\PY{l+m+mi}{20}\PY{p}{,} \PY{l+m+mi}{6}\PY{p}{)} 
        
        \PY{k+kn}{import} \PY{n+nn}{this}
\end{Verbatim}

    
    \begin{verbatim}
<IPython.core.display.HTML at 0x10603f9d0>
    \end{verbatim}

    
    \begin{Verbatim}[commandchars=\\\{\}]
Populating the interactive namespace from numpy and matplotlib
The Zen of Python, by Tim Peters

Beautiful is better than ugly.
Explicit is better than implicit.
Simple is better than complex.
Complex is better than complicated.
Flat is better than nested.
Sparse is better than dense.
Readability counts.
Special cases aren't special enough to break the rules.
Although practicality beats purity.
Errors should never pass silently.
Unless explicitly silenced.
In the face of ambiguity, refuse the temptation to guess.
There should be one-- and preferably only one --obvious way to do it.
Although that way may not be obvious at first unless you're Dutch.
Now is better than never.
Although never is often better than *right* now.
If the implementation is hard to explain, it's a bad idea.
If the implementation is easy to explain, it may be a good idea.
Namespaces are one honking great idea -- let's do more of those!
    \end{Verbatim}

    \subsection{Introduction to IPython Notebook}

\includegraphics{static/img/eon.png}
\includegraphics{static/img/pyconza.png}

Tobie Nortje, \href{http://www.eon.co.za}{www.eon.co.za},
\href{http://twitter.com/tooblippe}{@tooblippe}

tobie.nortje@eon.co.za

http://bit.ly/pyconza-notebook

    \section{ls}

\begin{itemize}
\itemsep1pt\parskip0pt\parsep0pt
\item
  Background and Notebook basics
\item
  Plotting
\item
  Symbolic Mathematics and Typesetting
\item
  Pandas and the Pandas Dataframe
\item
  Quick Machine Learning example
\item
  Publishing Your Work
\end{itemize}

    \section{whoami}

\begin{itemize}
\item
  Electrical Engineer / Hacker
\item
  Currently busy with M.Eng
\item
  Principal Consultant at
  \href{http://www.eon.co.za/index.php/our-services-main/our-services/business-analytics}{Eon
  Consulting}
\item
  Management consultant focus on research, analytics
\item
  Clients Eskom, City Power, Ekhurhuleni, Neotel
\item
  Use Windows, Ubuntu GNU/Linux, Mac, IPhone and Android
\item
  People that I work with uses Excel (\emph{alot}) {[}\textbf{big
  time}{]}
\item
  Hack with Python, mySQL, Arduino, Energy Loggers, RaspberryPi
\item
  You will see Python 2.7 here. Almost time to upgrade
\end{itemize}

    \section{Python is Popular}

\begin{figure}[htbp]
\centering
\includegraphics{static/img/python_companies2.png}
\end{figure}

    \section{Practical Use}

\begin{figure}[htbp]
\centering
\includegraphics{static/img/hotwater.png}
\end{figure}

    \section{Why Python for Analytics and Visualization?}

10 years ago, Python was considered exotic in the analytics space -- at
best. Languages/packages like R and Matlab dominated the scene. Today,
Python has become a major force in data analytics \& visualization due
to a number of characteristics: * \textbf{Multi-purpose}: prototyping,
development, production, systems admin -- one for all *
\textbf{Libraries}: there is a library for almost any task or problem
you face * \textbf{Efficiency}: speeds up IT development tasks for
analytics applications * \textbf{Performance}: Python has evolved from a
scripting language to a `meta' language with bridges to all high
performance environments (e.g.~multi-core CPUs, GPUs, clusters) *
\textbf{Interoperability}: Python integrates with almost any other
language/ technology * \textbf{Interactivity}: Python allows domain
experts to get closer to their business and financial data pools and to
do real-time analytics * \textbf{Collaboration}: solutions like Wakari
with IPython Notebook allow the easy sharing of code, data, results,
graphics, etc.

    \subsection{The Python Science Stack}

\begin{itemize}
\item
  \textbf{Python} -- the Python interpreter itself
\item
  \textbf{NumPy} -- high performance, flexible array structures and
  operations
\item
  \textbf{SciPy} -- scientific modules and functions (regression,
  optimization, integration)
\item
  \textbf{pandas} -- time series and panel data analysis and I/O
\item
  \textbf{PyTables} -- hierarchical, high performance database (e.g.~for
  out-of-memory analytics)
\item
  \textbf{matplotlib} -- 2d and 3d visualization
\item
  \textbf{IPython} -- interactive data analytics, visualization,
  publishing
\item
  The list is growing\ldots{}
\end{itemize}

\begin{figure}[htbp]
\centering
\includegraphics{static/img/scistack.png}
\caption{SciPy}
\end{figure}

    \subsection{IPython Introduction \includegraphics{static/img/ipy.png}}

IPython provides a rich architecture for interactive computing with: *
Powerful interactive shells (terminal and Qt-based) * Browser-based
notebook support for code, text, math expressions, inline plots *
Support for interactive data visualization and use of GUI toolkits *
Easy to use, high performance tools for parallel computing

The main reasons I have been using it includes: * Plotting is possible
in the QT console or the Notebook * Magic functions makes life easier
(magics gets called with a \%) * I also use it as a replacement shell
for Windows Shell or Terminal * Code Completion * And of course\ldots{}
\textbf{Data Analysis}

    \section{Notebook Basics}

The IPython Notebook is a web-based interactive computational
environment where you can: * Combine code execution * Text * Mathematics
* Plots and rich media into a single document * Used to teach classes
(Berkley), talks, publish papers etc. It also features: * Code
Completion * Help and Docstrings * Markdown cells for composing
documents * Run it locally or on any webserver with Python installed

\paragraph{Everything you see here is standard Python and Markdown code
running in a browser on top of an IPython kernel using Python 2.7 `}

    \subsection{Some Helpful Commands}

\begin{itemize}
\itemsep1pt\parskip0pt\parsep0pt
\item
  \textbf{We are now live in an IPython Notebook sessions!}
\end{itemize}

\textbf{Command} \textbar{} \textbf{Description} \textbar{}
------------\textbar{}----------- \textbar{} \textbar{} ? \textbar{}
Overview of IPython's features \%quickref \textbar{} Quick reference.
help \textbar{} Python's own help system. object? \textbar{} Inspect an
object

\begin{figure}[htbp]
\centering
\includegraphics{static/img/terminal.png}
\caption{terminal}
\end{figure}

    \section{Let's Get Started}

\begin{itemize}
\itemsep1pt\parskip0pt\parsep0pt
\item
  Each cell is populated with Markdown or Python Code
\item
  This is a markdown cell (Double Click To Reveal)
\item
  The notebook is currently in presentation mode
\item
  Running a cell
\item
  \emph{SHIFT+ENTER} will run the contents of a cell and move to the
  next one
\item
  \emph{CTRL+ENTER} run the cell in place and don't move to the next
  cell
\item
  Help
\item
  \emph{CTRL-m h} show keyboard shortcuts
\item
  Save
\item
  At any point, even when the Kernel is busy, you can press
  \emph{CTRL-S} to save the notebook
\end{itemize}

    \begin{Verbatim}[commandchars=\\\{\}]
{\color{incolor}In [{\color{incolor}}]:} \PY{c}{\PYZsh{} press shift\PYZhy{}enter to run code}
       
       \PY{c}{\PYZsh{} Create and Set Variable a to the value of 4}
       \PY{n}{a} \PY{o}{=} \PY{l+m+mi}{4}
       
       \PY{c}{\PYZsh{} Create and Set Variable b to the value of 2}
       \PY{n}{b} \PY{o}{=} \PY{l+m+mi}{3}
       
       \PY{k}{print} \PY{l+s}{\PYZdq{}}\PY{l+s}{Hallo PyConZA 2014}\PY{l+s}{\PYZdq{}}
       \PY{k}{print} \PY{l+s}{\PYZdq{}}\PY{l+s}{a + b =}\PY{l+s}{\PYZdq{}}\PY{p}{,} \PY{n}{a}\PY{o}{+}\PY{n}{b}
       
       \PY{k}{print} \PY{l+s}{\PYZdq{}}\PY{l+s}{Now this is weird a/b = }\PY{l+s}{\PYZdq{}}\PY{p}{,} \PY{n}{a}\PY{o}{/}\PY{n}{b}
       
       \PY{n}{b} \PY{o}{=} \PY{l+m+mf}{3.0} \PY{c}{\PYZsh{}\PYZlt{}\PYZhy{}\PYZhy{}\PYZhy{}\PYZhy{} Be Carefull}
       
       \PY{k}{print} \PY{l+s}{\PYZdq{}}\PY{l+s}{This is even weirder a/b = }\PY{l+s}{\PYZdq{}}\PY{p}{,} \PY{n}{a}\PY{o}{/}\PY{n}{b}
\end{Verbatim}

    \subsection{Using the Help System}

\begin{itemize}
\itemsep1pt\parskip0pt\parsep0pt
\item
  The \texttt{\%quickref} command can be used to obtain a bit more
  information
\end{itemize}

    \begin{Verbatim}[commandchars=\\\{\}]
{\color{incolor}In [{\color{incolor}}]:} \PY{c}{\PYZsh{}IPython \PYZhy{}\PYZhy{} An enhanced Interactive Python \PYZhy{} Quick Reference Card}
       \PY{o}{\PYZpc{}}\PY{k}{quickref}  \PY{c}{\PYZsh{} now press shift\PYZhy{}enter}
\end{Verbatim}

    \subsection{Code Completion and Introspection}

\begin{itemize}
\itemsep1pt\parskip0pt\parsep0pt
\item
  The next cell defines a function with a bit of a long name
\item
  By using the \texttt{?} command the docstring can we viewed
\item
  \texttt{??} will open up the source code
\item
  The autocomplete function is also demonstrated
\end{itemize}

    \begin{Verbatim}[commandchars=\\\{\}]
{\color{incolor}In [{\color{incolor}}]:} \PY{c}{\PYZsh{} lets degine a function with a long name.}
       \PY{k}{def} \PY{n+nf}{long\PYZus{}silly\PYZus{}dummy\PYZus{}name}\PY{p}{(}\PY{n}{a}\PY{p}{,} \PY{n}{b}\PY{p}{)}\PY{p}{:}
           \PY{l+s+sd}{\PYZdq{}\PYZdq{}\PYZdq{}}
       \PY{l+s+sd}{    This is the docstring for this function }
       \PY{l+s+sd}{    It takes two arguments a and b}
       \PY{l+s+sd}{    It returns the sum of a+5 and b*5}
       \PY{l+s+sd}{    No error checking is done!}
       \PY{l+s+sd}{    \PYZdq{}\PYZdq{}\PYZdq{}}
           \PY{n}{a} \PY{o}{\PYZhy{}}\PY{o}{=} \PY{l+m+mi}{5}
           \PY{n}{b} \PY{o}{*}\PY{o}{=} \PY{l+m+mi}{5}
           \PY{k}{return} \PY{n}{a}\PY{o}{+}\PY{n}{b}
       
       
       \PY{k}{def} \PY{n+nf}{long\PYZus{}silly\PYZus{}dummy\PYZus{}name\PYZus{}2}\PY{p}{(}\PY{n}{a}\PY{p}{,} \PY{n}{b}\PY{p}{)}\PY{p}{:}
           \PY{l+s+sd}{\PYZdq{}\PYZdq{}\PYZdq{}}
       \PY{l+s+sd}{    This is the docstring for this function }
       \PY{l+s+sd}{    It takes two arguments a and b}
       \PY{l+s+sd}{    It returns the sum of a+5 and b*5}
       \PY{l+s+sd}{    No error checking is done!}
       \PY{l+s+sd}{    \PYZdq{}\PYZdq{}\PYZdq{}}
           \PY{n}{a} \PY{o}{+}\PY{o}{=} \PY{l+m+mi}{5}
           \PY{n}{b} \PY{o}{*}\PY{o}{=} \PY{l+m+mi}{5}
           \PY{k}{return} \PY{n}{a}\PY{o}{+}\PY{n}{b}
       
       \PY{c}{\PYZsh{} again we press SHIFT\PYZhy{}Enter}
       \PY{c}{\PYZsh{} this will run the function and add it to the namespace}
       
       \PY{k}{for} \PY{n}{i} \PY{o+ow}{in} \PY{n+nb}{dir}\PY{p}{(}\PY{p}{)}\PY{p}{:}
           \PY{k}{if} \PY{l+s}{\PYZdq{}}\PY{l+s}{silly}\PY{l+s}{\PYZdq{}} \PY{o+ow}{in} \PY{n}{i}\PY{p}{:} 
               \PY{k}{print} \PY{n}{i}
\end{Verbatim}

    \begin{Verbatim}[commandchars=\\\{\}]
{\color{incolor}In [{\color{incolor}}]:} \PY{c}{\PYZsh{} lets get the docstring or some help}
       \PY{n}{long\PYZus{}silly\PYZus{}dummy\PYZus{}name}\PY{err}{?}
       
       \PY{c}{\PYZsh{} This will open a tab at the bottom of the web page}
       \PY{c}{\PYZsh{} You can close it by clicking on the small cross on the right}
\end{Verbatim}

    \begin{Verbatim}[commandchars=\\\{\}]
{\color{incolor}In [{\color{incolor}}]:} \PY{c}{\PYZsh{}view the source}
       \PY{n}{long\PYZus{}silly\PYZus{}dummy\PYZus{}name}\PY{err}{?}\PY{err}{?}
\end{Verbatim}

    \begin{Verbatim}[commandchars=\\\{\}]
{\color{incolor}In [{\color{incolor}}]:} \PY{c}{\PYZsh{}press tab to autocplete}
       \PY{n}{long\PYZus{}silly\PYZus{}dummy\PYZus{}name\PYZus{}2}
\end{Verbatim}

    \begin{Verbatim}[commandchars=\\\{\}]
{\color{incolor}In [{\color{incolor}}]:} \PY{c}{\PYZsh{} press shift\PYZhy{}enter to run}
       \PY{n}{long\PYZus{}silly\PYZus{}dummy\PYZus{}name}\PY{p}{(}\PY{l+m+mi}{5}\PY{p}{,}\PY{l+m+mi}{6}\PY{p}{)}
\end{Verbatim}

    \subsection{Using markdown}

\begin{itemize}
\itemsep1pt\parskip0pt\parsep0pt
\item
  You can set the contents type of a cell in the toolbar above.
\item
  When Markdown is selected you can enter markdown in a cell and its
  contents will be rendered as HTML.
\end{itemize}

\begin{enumerate}
\def\labelenumi{\arabic{enumi}.}
\itemsep1pt\parskip0pt\parsep0pt
\item
  The markdown syntax can be
  \href{http://daringfireball.net/projects/markdown/}{found here}
\end{enumerate}

\section{This is heading 1}

\subsection{This is heading 2}

\subsubsection{This is heading 5}

\begin{longtable}[c]{@{}ll@{}}
\hline\noalign{\medskip}
IPYTHON & PyConZA
\\\noalign{\medskip}
\hline\noalign{\medskip}
\includegraphics{static/img/ipy.png} &
\includegraphics{static/img/pyconza.png}
\\\noalign{\medskip}
\hline
\end{longtable}

\begin{quote}
Beautiful is better than ugly. Explicit is better than implicit. Simple
is better than complex. Complex is better than complicated.
\end{quote}

    \subsection{IPython and Notebook Magics}

\begin{itemize}
\itemsep1pt\parskip0pt\parsep0pt
\item
  IPython has a set of predefined `magic functions' that you can call
  with a command line style syntax
\item
  Think of them as little helper macro's/funcions
\item
  There are two kinds of magics, line-oriented and cell-oriented
\item
  \textbf{\emph{Line magics}} are prefixed with the \texttt{\%}
  character and work much like OS command-line calls: they get as an
  argument the rest of the line, where arguments are passed without
  parentheses or quotes.
\item
  \textbf{\emph{Cell magics}} are prefixed with a double \texttt{\%\%},
  and they are functions that get as an argument not only the rest of
  the line, but also the lines below it in a separate argument
\end{itemize}

    \subsection{timeit magic}

\begin{itemize}
\itemsep1pt\parskip0pt\parsep0pt
\item
  The \texttt{\%\%timeit} magic can be used to evaluate the average time
  your loop or piece of code is taking to complete
\end{itemize}

    \begin{Verbatim}[commandchars=\\\{\}]
{\color{incolor}In [{\color{incolor}}]:} \PY{o}{\PYZpc{}\PYZpc{}}\PY{k}{timeit}
       \PY{n}{x} \PY{o}{=} \PY{l+m+mi}{0}   \PY{c}{\PYZsh{} setup}
       \PY{k}{for} \PY{n}{i} \PY{o+ow}{in} \PY{n+nb}{xrange}\PY{p}{(}\PY{l+m+mi}{100000}\PY{p}{)}\PY{p}{:}    
           \PY{n}{x} \PY{o}{=} \PY{n}{x} \PY{o}{+} \PY{n}{i}\PY{o}{*}\PY{o}{*}\PY{l+m+mi}{2}
           \PY{n}{x} \PY{o}{=} \PY{n}{x} \PY{o}{+} \PY{n}{i}\PY{o}{*}\PY{o}{*}\PY{l+m+mi}{2}
           
\end{Verbatim}

    \begin{Verbatim}[commandchars=\\\{\}]
{\color{incolor}In [{\color{incolor}}]:} \PY{o}{\PYZpc{}\PYZpc{}}\PY{k}{timeit}
       \PY{n}{x} \PY{o}{=} \PY{l+m+mi}{0}   \PY{c}{\PYZsh{} setup\PYZsh{}}
       \PY{k}{for} \PY{n}{i} \PY{o+ow}{in} \PY{n+nb}{xrange}\PY{p}{(}\PY{l+m+mi}{100000}\PY{p}{)}\PY{p}{:}   
           \PY{n}{x} \PY{o}{+}\PY{o}{=} \PY{n}{i}\PY{o}{*}\PY{o}{*}\PY{l+m+mi}{2}
\end{Verbatim}

    \subsection{Running Shell Commands}

\begin{itemize}
\itemsep1pt\parskip0pt\parsep0pt
\item
  I now use IPython as my default shell scripting language
\item
  Example - put the contents of the current directory into a list and
  count the file types
\item
  The \texttt{!} before a command indicates that you want to run a
  system command.
\end{itemize}

    \begin{Verbatim}[commandchars=\\\{\}]
{\color{incolor}In [{\color{incolor}2}]:} \PY{n}{filelist} \PY{o}{=} \PY{err}{!}\PY{n}{ls} \PY{n}{static}\PY{o}{/}\PY{n}{img}                    \PY{c}{\PYZsh{}read the current directory into variable}
        
        
        \PY{n}{d} \PY{o}{=}\PY{p}{\PYZob{}}\PY{p}{\PYZcb{}}
        
        \PY{k}{for} \PY{n}{x}\PY{p}{,}\PY{n}{i} \PY{o+ow}{in} \PY{n+nb}{enumerate}\PY{p}{(}\PY{n}{filelist}\PY{p}{)}\PY{p}{:}
            
            \PY{n}{key} \PY{o}{=} \PY{n}{i}\PY{p}{[} \PY{n}{i}\PY{o}{.}\PY{n}{find}\PY{p}{(}\PY{l+s}{\PYZdq{}}\PY{l+s}{.}\PY{l+s}{\PYZdq{}}\PY{p}{)}\PY{o}{+}\PY{l+m+mi}{1}\PY{p}{:}\PY{n+nb}{len}\PY{p}{(}\PY{n}{i}\PY{p}{)} \PY{p}{]}
            
            \PY{k}{if} \PY{n}{d}\PY{o}{.}\PY{n}{has\PYZus{}key}\PY{p}{(}\PY{n}{key}\PY{p}{)}\PY{p}{:}
                \PY{n}{d}\PY{p}{[} \PY{n}{i}\PY{p}{[}\PY{n}{i}\PY{o}{.}\PY{n}{find}\PY{p}{(}\PY{l+s}{\PYZdq{}}\PY{l+s}{.}\PY{l+s}{\PYZdq{}}\PY{p}{)}\PY{o}{+}\PY{l+m+mi}{1}\PY{p}{:}\PY{n+nb}{len}\PY{p}{(}\PY{n}{i}\PY{p}{)}\PY{p}{]} \PY{p}{]} \PY{o}{+}\PY{o}{=} \PY{l+m+mi}{1}
                
            \PY{k}{else}\PY{p}{:}
                \PY{n}{d}\PY{p}{[}\PY{n}{i}\PY{p}{[}\PY{n}{i}\PY{o}{.}\PY{n}{find}\PY{p}{(}\PY{l+s}{\PYZdq{}}\PY{l+s}{.}\PY{l+s}{\PYZdq{}}\PY{p}{)}\PY{o}{+}\PY{l+m+mi}{1}\PY{p}{:}\PY{n+nb}{len}\PY{p}{(}\PY{n}{i}\PY{p}{)}\PY{p}{]}\PY{p}{]} \PY{o}{=} \PY{l+m+mi}{1}
            
        
        \PY{k}{for} \PY{n}{key} \PY{o+ow}{in} \PY{n}{d}\PY{p}{:}
            \PY{k}{print} \PY{n}{key}\PY{p}{,} \PY{n}{d}\PY{p}{[}\PY{n}{key}\PY{p}{]}
\end{Verbatim}

    \begin{Verbatim}[commandchars=\\\{\}]
pptx 1
json 1
db 1
jpg 1
png 16
    \end{Verbatim}


    \subsection{\subsection{Embedding Images * Images can be added using Python code *
Or by adding inline HTML * This slide shows inline html (double click to
show) * The next slide will use Python syntax to add it }}


    \begin{Verbatim}[commandchars=\\\{\}]
{\color{incolor}In [{\color{incolor}}]:} \PY{k+kn}{from} \PY{n+nn}{IPython.display} \PY{k+kn}{import} \PY{n}{Image}
       \PY{n}{Image}\PY{p}{(}\PY{l+s}{\PYZsq{}}\PY{l+s}{static/img/pyconza.png}\PY{l+s}{\PYZsq{}}\PY{p}{)}
\end{Verbatim}

    \subsection{Adding YouYube videos}

\begin{itemize}
\itemsep1pt\parskip0pt\parsep0pt
\item
  You can also add rich media like YouTube Videos into the notebook
\end{itemize}

    \begin{Verbatim}[commandchars=\\\{\}]
{\color{incolor}In [{\color{incolor}}]:} \PY{c}{\PYZsh{}you can embed the HTML or programatically add it.}
       \PY{k+kn}{from} \PY{n+nn}{IPython.display} \PY{k+kn}{import} \PY{n}{YouTubeVideo}
       \PY{n}{YouTubeVideo}\PY{p}{(}\PY{l+s}{\PYZsq{}}\PY{l+s}{iwVvqwLDsJo}\PY{l+s}{\PYZsq{}}\PY{p}{,} \PY{n}{width}\PY{o}{=}\PY{l+m+mi}{1000}\PY{p}{,} \PY{n}{height}\PY{o}{=}\PY{l+m+mi}{600}\PY{p}{)}
\end{Verbatim}

    \subsection{Plotting with Matplotlib}

\begin{itemize}
\itemsep1pt\parskip0pt\parsep0pt
\item
  Matplotlib is a Python 2D plotting library
\item
  Produces publication quality figures
\item
  Variety of hardcopy formats and interactive environments across
  platforms
\item
  Matplotlib can be used in
\item
  Python scripts,
\item
  The Python and IPython shell,
\item
  Web application servers,
\item
  Graphical user interface toolkits.
\end{itemize}

    \begin{Verbatim}[commandchars=\\\{\}]
{\color{incolor}In [{\color{incolor}3}]:} \PY{k+kn}{from} \PY{n+nn}{matplotlib.pylab} \PY{k+kn}{import} \PY{n}{xkcd}
        \PY{k+kn}{from} \PY{n+nn}{numpy} \PY{k+kn}{import} \PY{o}{*}
        
        \PY{c}{\PYZsh{}generate some data}
        \PY{n}{n} \PY{o}{=} \PY{n}{array}\PY{p}{(}\PY{p}{[}\PY{l+m+mi}{0}\PY{p}{,}\PY{l+m+mi}{1}\PY{p}{,}\PY{l+m+mi}{2}\PY{p}{,}\PY{l+m+mi}{3}\PY{p}{,}\PY{l+m+mi}{4}\PY{p}{,}\PY{l+m+mi}{5}\PY{p}{]}\PY{p}{)}
        \PY{n}{xx} \PY{o}{=} \PY{n}{np}\PY{o}{.}\PY{n}{linspace}\PY{p}{(}\PY{o}{\PYZhy{}}\PY{l+m+mf}{0.75}\PY{p}{,} \PY{l+m+mf}{1.}\PY{p}{,} \PY{l+m+mi}{100}\PY{p}{)}
        \PY{n}{x} \PY{o}{=} \PY{n}{linspace}\PY{p}{(}\PY{l+m+mi}{0}\PY{p}{,} \PY{l+m+mi}{5}\PY{p}{,} \PY{l+m+mi}{10}\PY{p}{)}
        \PY{n}{y} \PY{o}{=} \PY{n}{x} \PY{o}{*}\PY{o}{*} \PY{l+m+mi}{2}
        
        \PY{n}{fig}\PY{p}{,} \PY{n}{axes} \PY{o}{=} \PY{n}{plt}\PY{o}{.}\PY{n}{subplots}\PY{p}{(}\PY{l+m+mi}{1}\PY{p}{,} \PY{l+m+mi}{4}\PY{p}{,} \PY{n}{figsize}\PY{o}{=}\PY{p}{(}\PY{l+m+mi}{12}\PY{p}{,}\PY{l+m+mi}{3}\PY{p}{)}\PY{p}{)}
        
        \PY{n}{axes}\PY{p}{[}\PY{l+m+mi}{0}\PY{p}{]}\PY{o}{.}\PY{n}{scatter}\PY{p}{(}\PY{n}{xx}\PY{p}{,} \PY{n}{xx} \PY{o}{+} \PY{l+m+mf}{0.25}\PY{o}{*}\PY{n}{randn}\PY{p}{(}\PY{n+nb}{len}\PY{p}{(}\PY{n}{xx}\PY{p}{)}\PY{p}{)}\PY{p}{)}
        \PY{n}{axes}\PY{p}{[}\PY{l+m+mi}{0}\PY{p}{]}\PY{o}{.}\PY{n}{set\PYZus{}title}\PY{p}{(}\PY{l+s}{\PYZsq{}}\PY{l+s}{scatter}\PY{l+s}{\PYZsq{}}\PY{p}{)}
        \PY{n}{axes}\PY{p}{[}\PY{l+m+mi}{1}\PY{p}{]}\PY{o}{.}\PY{n}{step}\PY{p}{(}\PY{n}{n}\PY{p}{,} \PY{n}{n}\PY{o}{*}\PY{o}{*}\PY{l+m+mi}{2}\PY{p}{,} \PY{n}{lw}\PY{o}{=}\PY{l+m+mi}{2}\PY{p}{)}
        \PY{n}{axes}\PY{p}{[}\PY{l+m+mi}{1}\PY{p}{]}\PY{o}{.}\PY{n}{set\PYZus{}title}\PY{p}{(}\PY{l+s}{\PYZsq{}}\PY{l+s}{step}\PY{l+s}{\PYZsq{}}\PY{p}{)}
        \PY{n}{axes}\PY{p}{[}\PY{l+m+mi}{2}\PY{p}{]}\PY{o}{.}\PY{n}{bar}\PY{p}{(}\PY{n}{n}\PY{p}{,} \PY{n}{n}\PY{o}{*}\PY{o}{*}\PY{l+m+mi}{2}\PY{p}{,} \PY{n}{align}\PY{o}{=}\PY{l+s}{\PYZdq{}}\PY{l+s}{center}\PY{l+s}{\PYZdq{}}\PY{p}{,} \PY{n}{width}\PY{o}{=}\PY{l+m+mf}{0.5}\PY{p}{,} \PY{n}{alpha}\PY{o}{=}\PY{l+m+mf}{0.5}\PY{p}{)}
        \PY{n}{axes}\PY{p}{[}\PY{l+m+mi}{2}\PY{p}{]}\PY{o}{.}\PY{n}{set\PYZus{}title}\PY{p}{(}\PY{l+s}{\PYZsq{}}\PY{l+s}{bar}\PY{l+s}{\PYZsq{}}\PY{p}{)}
        \PY{n}{axes}\PY{p}{[}\PY{l+m+mi}{3}\PY{p}{]}\PY{o}{.}\PY{n}{fill\PYZus{}between}\PY{p}{(}\PY{n}{x}\PY{p}{,} \PY{n}{x}\PY{o}{*}\PY{o}{*}\PY{l+m+mi}{2}\PY{p}{,} \PY{n}{x}\PY{o}{*}\PY{o}{*}\PY{l+m+mi}{3}\PY{p}{,} \PY{n}{color}\PY{o}{=}\PY{l+s}{\PYZdq{}}\PY{l+s}{green}\PY{l+s}{\PYZdq{}}\PY{p}{,} \PY{n}{alpha}\PY{o}{=}\PY{l+m+mf}{0.5}\PY{p}{)}\PY{p}{;}
        \PY{n}{axes}\PY{p}{[}\PY{l+m+mi}{3}\PY{p}{]}\PY{o}{.}\PY{n}{set\PYZus{}title}\PY{p}{(}\PY{l+s}{\PYZsq{}}\PY{l+s}{fill}\PY{l+s}{\PYZsq{}}\PY{p}{)}
        
        \PY{k}{for} \PY{n}{i} \PY{o+ow}{in} \PY{n+nb}{range}\PY{p}{(}\PY{l+m+mi}{4}\PY{p}{)}\PY{p}{:}
            \PY{n}{axes}\PY{p}{[}\PY{n}{i}\PY{p}{]}\PY{o}{.}\PY{n}{set\PYZus{}xlabel}\PY{p}{(}\PY{l+s}{\PYZsq{}}\PY{l+s}{x}\PY{l+s}{\PYZsq{}}\PY{p}{)}
        \PY{n}{axes}\PY{p}{[}\PY{l+m+mi}{0}\PY{p}{]}\PY{o}{.}\PY{n}{set\PYZus{}ylabel}\PY{p}{(}\PY{l+s}{\PYZsq{}}\PY{l+s}{y}\PY{l+s}{\PYZsq{}}\PY{p}{)}
        \PY{n}{show}\PY{p}{(}\PY{p}{)}
\end{Verbatim}

    \begin{center}
    \adjustimage{max size={0.9\linewidth}{0.9\paperheight}}{demo_files/demo_34_0.png}
    \end{center}
    { \hspace*{\fill} \\}
    
    \subsection{Combined plots}

\begin{itemize}
\itemsep1pt\parskip0pt\parsep0pt
\item
  Matplotlib allows plots to be combined
\end{itemize}

    \begin{Verbatim}[commandchars=\\\{\}]
{\color{incolor}In [{\color{incolor}}]:} \PY{c}{\PYZsh{}CustomPlot()}
       \PY{n}{font\PYZus{}size} \PY{o}{=} \PY{l+m+mi}{20}
       \PY{n}{figsize}\PY{p}{(}\PY{l+m+mf}{11.5}\PY{p}{,} \PY{l+m+mi}{5}\PY{p}{)} 
       \PY{n}{fig}\PY{p}{,} \PY{n}{ax} \PY{o}{=} \PY{n}{plt}\PY{o}{.}\PY{n}{subplots}\PY{p}{(}\PY{p}{)}
       
       \PY{n}{ax}\PY{o}{.}\PY{n}{plot}\PY{p}{(}\PY{n}{xx}\PY{p}{,} \PY{n}{xx}\PY{o}{*}\PY{o}{*}\PY{l+m+mi}{2}\PY{p}{,} \PY{n}{xx}\PY{p}{,} \PY{n}{xx}\PY{o}{*}\PY{o}{*}\PY{l+m+mi}{3}\PY{p}{)}
       \PY{n}{ax}\PY{o}{.}\PY{n}{set\PYZus{}title}\PY{p}{(}\PY{l+s}{r\PYZdq{}}\PY{l+s}{Combined Plot \PYZdl{}y=x\PYZca{}2\PYZdl{} vs. \PYZdl{}y=x\PYZca{}3\PYZdl{}}\PY{l+s}{\PYZdq{}}\PY{p}{,} \PY{n}{fontsize} \PY{o}{=} \PY{n}{font\PYZus{}size}\PY{p}{)}
       \PY{n}{ax}\PY{o}{.}\PY{n}{set\PYZus{}xlabel}\PY{p}{(}\PY{l+s}{r\PYZsq{}}\PY{l+s}{\PYZdl{}x\PYZdl{}}\PY{l+s}{\PYZsq{}}\PY{p}{,} \PY{n}{fontsize} \PY{o}{=} \PY{n}{font\PYZus{}size}\PY{p}{)}
       \PY{n}{ax}\PY{o}{.}\PY{n}{set\PYZus{}ylabel}\PY{p}{(}\PY{l+s}{r\PYZsq{}}\PY{l+s}{\PYZdl{}y\PYZdl{}}\PY{l+s}{\PYZsq{}}\PY{p}{,} \PY{n}{fontsize} \PY{o}{=} \PY{n}{font\PYZus{}size}\PY{p}{)}
       \PY{n}{fig}\PY{o}{.}\PY{n}{tight\PYZus{}layout}\PY{p}{(}\PY{p}{)}
       \PY{c}{\PYZsh{} inset}
       \PY{n}{inset\PYZus{}ax} \PY{o}{=} \PY{n}{fig}\PY{o}{.}\PY{n}{add\PYZus{}axes}\PY{p}{(}\PY{p}{[}\PY{l+m+mf}{0.29}\PY{p}{,} \PY{l+m+mf}{0.45}\PY{p}{,} \PY{l+m+mf}{0.35}\PY{p}{,} \PY{l+m+mf}{0.35}\PY{p}{]}\PY{p}{)} \PY{c}{\PYZsh{} X, Y, width, height}
       \PY{n}{inset\PYZus{}ax}\PY{o}{.}\PY{n}{plot}\PY{p}{(}\PY{n}{xx}\PY{p}{,} \PY{n}{xx}\PY{o}{*}\PY{o}{*}\PY{l+m+mi}{2}\PY{p}{,} \PY{n}{xx}\PY{p}{,} \PY{n}{xx}\PY{o}{*}\PY{o}{*}\PY{l+m+mi}{3}\PY{p}{)}
       \PY{n}{inset\PYZus{}ax}\PY{o}{.}\PY{n}{set\PYZus{}title}\PY{p}{(}\PY{l+s}{r\PYZsq{}}\PY{l+s}{zoom \PYZdl{}x=0\PYZdl{}}\PY{l+s}{\PYZsq{}}\PY{p}{,}\PY{n}{fontsize}\PY{o}{=}\PY{n}{font\PYZus{}size}\PY{p}{)}
       \PY{c}{\PYZsh{} set axis range}
       \PY{n}{inset\PYZus{}ax}\PY{o}{.}\PY{n}{set\PYZus{}xlim}\PY{p}{(}\PY{o}{\PYZhy{}}\PY{o}{.}\PY{l+m+mi}{2}\PY{p}{,} \PY{o}{.}\PY{l+m+mi}{2}\PY{p}{)}
       \PY{n}{inset\PYZus{}ax}\PY{o}{.}\PY{n}{set\PYZus{}ylim}\PY{p}{(}\PY{o}{\PYZhy{}}\PY{o}{.}\PY{l+m+mo}{005}\PY{p}{,} \PY{o}{.}\PY{l+m+mo}{01}\PY{p}{)}
       \PY{c}{\PYZsh{} set axis tick locations}
       \PY{n}{inset\PYZus{}ax}\PY{o}{.}\PY{n}{set\PYZus{}yticks}\PY{p}{(}\PY{p}{[}\PY{l+m+mi}{0}\PY{p}{,} \PY{l+m+mf}{0.005}\PY{p}{,} \PY{l+m+mf}{0.01}\PY{p}{]}\PY{p}{)}
       \PY{n}{inset\PYZus{}ax}\PY{o}{.}\PY{n}{set\PYZus{}xticks}\PY{p}{(}\PY{p}{[}\PY{o}{\PYZhy{}}\PY{l+m+mf}{0.1}\PY{p}{,}\PY{l+m+mi}{0}\PY{p}{,}\PY{o}{.}\PY{l+m+mi}{1}\PY{p}{]}\PY{p}{)}\PY{p}{;}
       \PY{n}{show}\PY{p}{(}\PY{p}{)}
\end{Verbatim}

    \subsection{Adding text to a plot}

    \begin{Verbatim}[commandchars=\\\{\}]
{\color{incolor}In [{\color{incolor}5}]:} \PY{c}{\PYZsh{}CustomPlot()}
        \PY{n}{figsize}\PY{p}{(}\PY{l+m+mi}{22}\PY{p}{,} \PY{l+m+mi}{9}\PY{p}{)} 
        \PY{n}{font\PYZus{}size} \PY{o}{=} \PY{l+m+mi}{50}
        \PY{n}{fig}\PY{p}{,} \PY{n}{ax} \PY{o}{=} \PY{n}{plt}\PY{o}{.}\PY{n}{subplots}\PY{p}{(}\PY{p}{)}
        \PY{n}{ax}\PY{o}{.}\PY{n}{plot}\PY{p}{(}\PY{n}{xx}\PY{p}{,} \PY{n}{xx}\PY{o}{*}\PY{o}{*}\PY{l+m+mi}{2}\PY{p}{,} \PY{n}{xx}\PY{p}{,} \PY{n}{xx}\PY{o}{*}\PY{o}{*}\PY{l+m+mi}{3}\PY{p}{)}
        
        \PY{n}{ax}\PY{o}{.}\PY{n}{set\PYZus{}xlabel}\PY{p}{(}\PY{l+s}{r\PYZsq{}}\PY{l+s}{\PYZdl{}x\PYZdl{}}\PY{l+s}{\PYZsq{}}\PY{p}{,} \PY{n}{fontsize} \PY{o}{=} \PY{n}{font\PYZus{}size}\PY{p}{)}
        \PY{n}{ax}\PY{o}{.}\PY{n}{set\PYZus{}ylabel}\PY{p}{(}\PY{l+s}{r\PYZsq{}}\PY{l+s}{\PYZdl{}y\PYZdl{}}\PY{l+s}{\PYZsq{}}\PY{p}{,} \PY{n}{fontsize} \PY{o}{=} \PY{n}{font\PYZus{}size}\PY{p}{)}
        \PY{n}{ax}\PY{o}{.}\PY{n}{set\PYZus{}title}\PY{p}{(}\PY{l+s}{r\PYZdq{}}\PY{l+s}{Adding Text \PYZdl{}y=x\PYZca{}2\PYZdl{} vs. \PYZdl{}y=x\PYZca{}3\PYZdl{}}\PY{l+s}{\PYZdq{}}\PY{p}{,} \PY{n}{fontsize} \PY{o}{=} \PY{n}{font\PYZus{}size}\PY{p}{)}
        
        \PY{n}{ax}\PY{o}{.}\PY{n}{text}\PY{p}{(}\PY{l+m+mf}{0.15}\PY{p}{,} \PY{l+m+mf}{0.2}\PY{p}{,} \PY{l+s}{r\PYZdq{}}\PY{l+s}{\PYZdl{}y=x\PYZca{}2\PYZdl{}}\PY{l+s}{\PYZdq{}}\PY{p}{,} \PY{n}{fontsize}\PY{o}{=}\PY{n}{font\PYZus{}size}\PY{p}{,} \PY{n}{color}\PY{o}{=}\PY{l+s}{\PYZdq{}}\PY{l+s}{blue}\PY{l+s}{\PYZdq{}}\PY{p}{)}
        \PY{n}{ax}\PY{o}{.}\PY{n}{text}\PY{p}{(}\PY{l+m+mf}{0.65}\PY{p}{,} \PY{l+m+mf}{0.1}\PY{p}{,} \PY{l+s}{r\PYZdq{}}\PY{l+s}{\PYZdl{}y=x\PYZca{}3\PYZdl{}}\PY{l+s}{\PYZdq{}}\PY{p}{,} \PY{n}{fontsize}\PY{o}{=}\PY{n}{font\PYZus{}size}\PY{p}{,} \PY{n}{color}\PY{o}{=}\PY{l+s}{\PYZdq{}}\PY{l+s}{green}\PY{l+s}{\PYZdq{}}\PY{p}{)}\PY{p}{;}
\end{Verbatim}

    \begin{center}
    \adjustimage{max size={0.9\linewidth}{0.9\paperheight}}{demo_files/demo_38_0.png}
    \end{center}
    { \hspace*{\fill} \\}
    
    \subsection{xkcd style plotting}

\begin{itemize}
\itemsep1pt\parskip0pt\parsep0pt
\item
  \href{http://matplotlib.org/xkcd/examples/showcase/xkcd.html}{matplolib
  v1.3} now includes a setting to make plots resemble xkcd styles.
\end{itemize}

\begin{figure}[htbp]
\centering
\includegraphics{static/img/xkcd.png}
\end{figure}

    \begin{Verbatim}[commandchars=\\\{\}]
{\color{incolor}In [{\color{incolor}}]:} \PY{k+kn}{from} \PY{n+nn}{matplotlib} \PY{k+kn}{import} \PY{n}{pyplot} \PY{k}{as} \PY{n}{plt}
       \PY{k+kn}{import} \PY{n+nn}{numpy} \PY{k+kn}{as} \PY{n+nn}{np}
       
       \PY{n}{fsize} \PY{o}{=} \PY{l+m+mi}{30}
       
       \PY{k}{with} \PY{n}{plt}\PY{o}{.}\PY{n}{xkcd}\PY{p}{(}\PY{p}{)}\PY{p}{:}
           \PY{n}{fig} \PY{o}{=} \PY{n}{plt}\PY{o}{.}\PY{n}{figure}\PY{p}{(}\PY{p}{)}
           \PY{n}{ax} \PY{o}{=} \PY{n}{fig}\PY{o}{.}\PY{n}{add\PYZus{}subplot}\PY{p}{(}\PY{l+m+mi}{1}\PY{p}{,} \PY{l+m+mi}{1}\PY{p}{,} \PY{l+m+mi}{1}\PY{p}{)}
           \PY{n}{ax}\PY{o}{.}\PY{n}{spines}\PY{p}{[}\PY{l+s}{\PYZsq{}}\PY{l+s}{right}\PY{l+s}{\PYZsq{}}\PY{p}{]}\PY{o}{.}\PY{n}{set\PYZus{}color}\PY{p}{(}\PY{l+s}{\PYZsq{}}\PY{l+s}{none}\PY{l+s}{\PYZsq{}}\PY{p}{)}
           \PY{n}{ax}\PY{o}{.}\PY{n}{spines}\PY{p}{[}\PY{l+s}{\PYZsq{}}\PY{l+s}{top}\PY{l+s}{\PYZsq{}}\PY{p}{]}\PY{o}{.}\PY{n}{set\PYZus{}color}\PY{p}{(}\PY{l+s}{\PYZsq{}}\PY{l+s}{none}\PY{l+s}{\PYZsq{}}\PY{p}{)}
           \PY{n}{plt}\PY{o}{.}\PY{n}{xticks}\PY{p}{(}\PY{p}{[}\PY{p}{]}\PY{p}{)}
           \PY{n}{plt}\PY{o}{.}\PY{n}{yticks}\PY{p}{(}\PY{p}{[}\PY{p}{]}\PY{p}{)}
           \PY{n}{ax}\PY{o}{.}\PY{n}{set\PYZus{}ylim}\PY{p}{(}\PY{p}{[}\PY{o}{\PYZhy{}}\PY{l+m+mi}{30}\PY{p}{,} \PY{l+m+mi}{10}\PY{p}{]}\PY{p}{)}
       
           \PY{n}{data} \PY{o}{=} \PY{n}{np}\PY{o}{.}\PY{n}{ones}\PY{p}{(}\PY{l+m+mi}{100}\PY{p}{)}
           \PY{n}{data}\PY{p}{[}\PY{l+m+mi}{70}\PY{p}{:}\PY{p}{]} \PY{o}{\PYZhy{}}\PY{o}{=} \PY{n}{np}\PY{o}{.}\PY{n}{arange}\PY{p}{(}\PY{l+m+mi}{30}\PY{p}{)}
       
           \PY{n}{plt}\PY{o}{.}\PY{n}{annotate}\PY{p}{(}
               \PY{l+s}{\PYZsq{}}\PY{l+s}{THE DAY I REALIZED}\PY{l+s+se}{\PYZbs{}n}\PY{l+s}{I COULD COOK BACON}\PY{l+s+se}{\PYZbs{}n}\PY{l+s}{WHENEVER I WANTED}\PY{l+s}{\PYZsq{}}\PY{p}{,}
               \PY{n}{xy}\PY{o}{=}\PY{p}{(}\PY{l+m+mi}{70}\PY{p}{,} \PY{l+m+mi}{1}\PY{p}{)}\PY{p}{,} \PY{n}{arrowprops}\PY{o}{=}\PY{n+nb}{dict}\PY{p}{(}\PY{n}{arrowstyle}\PY{o}{=}\PY{l+s}{\PYZsq{}}\PY{l+s}{\PYZhy{}\PYZgt{}}\PY{l+s}{\PYZsq{}}\PY{p}{)}\PY{p}{,} \PY{n}{xytext}\PY{o}{=}\PY{p}{(}\PY{l+m+mi}{15}\PY{p}{,} \PY{o}{\PYZhy{}}\PY{l+m+mi}{10}\PY{p}{)}\PY{p}{,}\PY{n}{size}\PY{o}{=}\PY{n}{fsize}\PY{p}{)}
       
           \PY{n}{plt}\PY{o}{.}\PY{n}{plot}\PY{p}{(}\PY{n}{data}\PY{p}{)}
       
           \PY{n}{plt}\PY{o}{.}\PY{n}{xlabel}\PY{p}{(}\PY{l+s}{\PYZsq{}}\PY{l+s}{time}\PY{l+s}{\PYZsq{}}\PY{p}{,} \PY{n}{size}\PY{o}{=}\PY{n}{fsize}\PY{p}{)}
           \PY{n}{plt}\PY{o}{.}\PY{n}{ylabel}\PY{p}{(}\PY{l+s}{\PYZsq{}}\PY{l+s}{my overall health}\PY{l+s}{\PYZsq{}}\PY{p}{,} \PY{n}{size}\PY{o}{=}\PY{n}{fsize}\PY{p}{)}
       
           \PY{n}{fig} \PY{o}{=} \PY{n}{plt}\PY{o}{.}\PY{n}{figure}\PY{p}{(}\PY{p}{)}
           \PY{n}{ax} \PY{o}{=} \PY{n}{fig}\PY{o}{.}\PY{n}{add\PYZus{}subplot}\PY{p}{(}\PY{l+m+mi}{1}\PY{p}{,} \PY{l+m+mi}{1}\PY{p}{,} \PY{l+m+mi}{1}\PY{p}{)}
           \PY{n}{ax}\PY{o}{.}\PY{n}{bar}\PY{p}{(}\PY{p}{[}\PY{o}{\PYZhy{}}\PY{l+m+mf}{0.125}\PY{p}{,} \PY{l+m+mf}{1.0}\PY{o}{\PYZhy{}}\PY{l+m+mf}{0.125}\PY{p}{]}\PY{p}{,} \PY{p}{[}\PY{l+m+mi}{0}\PY{p}{,} \PY{l+m+mi}{100}\PY{p}{]}\PY{p}{,} \PY{l+m+mf}{0.25}\PY{p}{)}
           \PY{n}{ax}\PY{o}{.}\PY{n}{spines}\PY{p}{[}\PY{l+s}{\PYZsq{}}\PY{l+s}{right}\PY{l+s}{\PYZsq{}}\PY{p}{]}\PY{o}{.}\PY{n}{set\PYZus{}color}\PY{p}{(}\PY{l+s}{\PYZsq{}}\PY{l+s}{none}\PY{l+s}{\PYZsq{}}\PY{p}{)}
           \PY{n}{ax}\PY{o}{.}\PY{n}{spines}\PY{p}{[}\PY{l+s}{\PYZsq{}}\PY{l+s}{top}\PY{l+s}{\PYZsq{}}\PY{p}{]}\PY{o}{.}\PY{n}{set\PYZus{}color}\PY{p}{(}\PY{l+s}{\PYZsq{}}\PY{l+s}{none}\PY{l+s}{\PYZsq{}}\PY{p}{)}
           \PY{n}{ax}\PY{o}{.}\PY{n}{xaxis}\PY{o}{.}\PY{n}{set\PYZus{}ticks\PYZus{}position}\PY{p}{(}\PY{l+s}{\PYZsq{}}\PY{l+s}{bottom}\PY{l+s}{\PYZsq{}}\PY{p}{)}
           \PY{n}{ax}\PY{o}{.}\PY{n}{set\PYZus{}xticks}\PY{p}{(}\PY{p}{[}\PY{l+m+mi}{0}\PY{p}{,} \PY{l+m+mi}{1}\PY{p}{]}\PY{p}{)}
           \PY{n}{ax}\PY{o}{.}\PY{n}{set\PYZus{}xlim}\PY{p}{(}\PY{p}{[}\PY{o}{\PYZhy{}}\PY{l+m+mf}{0.5}\PY{p}{,} \PY{l+m+mf}{1.5}\PY{p}{]}\PY{p}{)}
           \PY{n}{ax}\PY{o}{.}\PY{n}{set\PYZus{}ylim}\PY{p}{(}\PY{p}{[}\PY{l+m+mi}{0}\PY{p}{,} \PY{l+m+mi}{110}\PY{p}{]}\PY{p}{)}
           \PY{n}{ax}\PY{o}{.}\PY{n}{set\PYZus{}xticklabels}\PY{p}{(}\PY{p}{[}\PY{l+s}{\PYZsq{}}\PY{l+s}{CONFIRMED BY}\PY{l+s+se}{\PYZbs{}n}\PY{l+s}{EXPERIMENT}\PY{l+s}{\PYZsq{}}\PY{p}{,} \PY{l+s}{\PYZsq{}}\PY{l+s}{REFUTED BY}\PY{l+s+se}{\PYZbs{}n}\PY{l+s}{EXPERIMENT}\PY{l+s}{\PYZsq{}}\PY{p}{]}\PY{p}{,}\PY{n}{size}\PY{o}{=}\PY{n}{fsize}\PY{p}{)}
           \PY{n}{plt}\PY{o}{.}\PY{n}{yticks}\PY{p}{(}\PY{p}{[}\PY{p}{]}\PY{p}{)}
       
           \PY{n}{plt}\PY{o}{.}\PY{n}{title}\PY{p}{(}\PY{l+s}{\PYZdq{}}\PY{l+s}{CLAIMS OF SUPERNATURAL POWERS}\PY{l+s}{\PYZdq{}}\PY{p}{,}\PY{n}{size}\PY{o}{=}\PY{n}{fsize}\PY{p}{)}
       
           \PY{n}{plt}\PY{o}{.}\PY{n}{show}\PY{p}{(}\PY{p}{)}
\end{Verbatim}

    \subsection{Symbolic math using SymPy}

\begin{itemize}
\itemsep1pt\parskip0pt\parsep0pt
\item
  SymPy is a Python library for symbolic mathematics.
\item
  It aims to become a full-featured computer algebra system (CAS) while
  keeping the code as simple as possible in order to be comprehensible
  and easily extensible.
\item
  SymPy is written entirely in Python and does not require any external
  libraries.
\end{itemize}

\begin{figure}[htbp]
\centering
\includegraphics{static/img/sympy.png}
\end{figure}

    \begin{Verbatim}[commandchars=\\\{\}]
{\color{incolor}In [{\color{incolor}}]:} \PY{k+kn}{from} \PY{n+nn}{sympy} \PY{k+kn}{import} \PY{o}{*}
       \PY{n}{init\PYZus{}printing}\PY{p}{(}\PY{n}{use\PYZus{}latex}\PY{o}{=}\PY{n+nb+bp}{True}\PY{p}{)}
       \PY{k+kn}{from} \PY{n+nn}{sympy} \PY{k+kn}{import} \PY{n}{solve}
       
       \PY{n}{x} \PY{o}{=} \PY{n}{Symbol}\PY{p}{(}\PY{l+s}{\PYZsq{}}\PY{l+s}{x}\PY{l+s}{\PYZsq{}}\PY{p}{)}
       \PY{n}{y} \PY{o}{=} \PY{n}{Symbol}\PY{p}{(}\PY{l+s}{\PYZsq{}}\PY{l+s}{y}\PY{l+s}{\PYZsq{}}\PY{p}{)}
       
       \PY{n}{series}\PY{p}{(}\PY{n}{exp}\PY{p}{(}\PY{n}{x}\PY{p}{)}\PY{p}{,} \PY{n}{x}\PY{p}{,} \PY{l+m+mi}{1}\PY{p}{,} \PY{l+m+mi}{5}\PY{p}{)}
\end{Verbatim}

    \begin{Verbatim}[commandchars=\\\{\}]
{\color{incolor}In [{\color{incolor}}]:} \PY{n}{eq} \PY{o}{=} \PY{p}{(}\PY{p}{(}\PY{n}{x}\PY{o}{+}\PY{n}{y}\PY{p}{)}\PY{o}{*}\PY{o}{*}\PY{l+m+mi}{2} \PY{o}{*} \PY{p}{(}\PY{n}{x}\PY{o}{+}\PY{l+m+mi}{1}\PY{p}{)}\PY{p}{)}
       
       \PY{n}{eq}
\end{Verbatim}

    \begin{Verbatim}[commandchars=\\\{\}]
{\color{incolor}In [{\color{incolor}}]:} \PY{n}{solve}\PY{p}{(}\PY{n}{eq}\PY{p}{)}
\end{Verbatim}

    \begin{Verbatim}[commandchars=\\\{\}]
{\color{incolor}In [{\color{incolor}}]:} \PY{n}{a} \PY{o}{=} \PY{l+m+mi}{1}\PY{o}{/}\PY{n}{x} \PY{o}{+} \PY{p}{(}\PY{n}{x}\PY{o}{*}\PY{n}{sin}\PY{p}{(}\PY{n}{x}\PY{p}{)} \PY{o}{\PYZhy{}} \PY{l+m+mi}{1}\PY{p}{)}\PY{o}{/}\PY{n}{x}
       
       \PY{n}{a}
\end{Verbatim}

    \begin{Verbatim}[commandchars=\\\{\}]
{\color{incolor}In [{\color{incolor}}]:} \PY{n}{simplify}\PY{p}{(}\PY{n}{a}\PY{p}{)}
\end{Verbatim}

    \begin{Verbatim}[commandchars=\\\{\}]
{\color{incolor}In [{\color{incolor}}]:} \PY{n}{integrate}\PY{p}{(}\PY{n}{x}\PY{o}{*}\PY{o}{*}\PY{l+m+mi}{2} \PY{o}{*} \PY{n}{exp}\PY{p}{(}\PY{n}{x}\PY{p}{)} \PY{o}{*} \PY{n}{cos}\PY{p}{(}\PY{n}{x}\PY{p}{)}\PY{p}{,} \PY{n}{x}\PY{p}{)}
\end{Verbatim}

    \subsection{Data Analysis Using the Pandas Library}

\begin{figure}[htbp]
\centering
\includegraphics{static/img/pandas.png}
\end{figure}

\begin{itemize}
\item
  pandas is a Python package providing fast, flexible, and expressive
  data structures designed to make working with ``relational'' or
  ``labeled'' data both easy and intuitive
\item
  The two primary data structures of pandas
\item
  Series (1-dimensional) and
\item
  DataFrame (2-dimensional) \textbf{Think Spreadhseet, Timeseries}
\end{itemize}

Handle the vast majority of typical use cases in finance, statistics,
social science, and many areas of engineering.

\begin{itemize}
\itemsep1pt\parskip0pt\parsep0pt
\item
  For R users, DataFrame provides everything that R's data.frame
  provides.
\item
  pandas is built on top of NumPy and is intended to integrate well
  within a scientific computing environment with many other 3rd party
  libraries.
\end{itemize}

    \begin{Verbatim}[commandchars=\\\{\}]
{\color{incolor}In [{\color{incolor}}]:} \PY{k+kn}{from} \PY{n+nn}{pandas} \PY{k+kn}{import} \PY{n}{DataFrame}\PY{p}{,} \PY{n}{read\PYZus{}csv}
       
       \PY{n}{Cape\PYZus{}Weather} \PY{o}{=} \PY{n}{DataFrame}\PY{p}{(} \PY{n}{read\PYZus{}csv}\PY{p}{(}\PY{l+s}{\PYZsq{}}\PY{l+s}{static/data/CapeTown\PYZus{}2009\PYZus{}Temperatures.csv}\PY{l+s}{\PYZsq{}} \PY{p}{)}\PY{p}{)}
       
       \PY{n}{Cape\PYZus{}Weather}\PY{o}{.}\PY{n}{head}\PY{p}{(}\PY{l+m+mi}{10}\PY{p}{)}
\end{Verbatim}

    \begin{Verbatim}[commandchars=\\\{\}]
{\color{incolor}In [{\color{incolor}}]:} \PY{c}{\PYZsh{}CustomPlot()}
       \PY{n}{figsize}\PY{p}{(}\PY{l+m+mi}{22}\PY{p}{,} \PY{l+m+mi}{10}\PY{p}{)}
       \PY{n}{font\PYZus{}size} \PY{o}{=} \PY{l+m+mi}{24}
       
       \PY{n}{title}\PY{p}{(}\PY{l+s}{\PYZsq{}}\PY{l+s}{Cape Town Temparature(2009)}\PY{l+s}{\PYZsq{}}\PY{p}{,}\PY{n}{fontsize} \PY{o}{=} \PY{n}{font\PYZus{}size}\PY{p}{)}
       \PY{n}{xlabel}\PY{p}{(}\PY{l+s}{\PYZsq{}}\PY{l+s}{Day number}\PY{l+s}{\PYZsq{}}\PY{p}{,}\PY{n}{fontsize} \PY{o}{=} \PY{n}{font\PYZus{}size}\PY{p}{)}
       \PY{n}{ylabel}\PY{p}{(}\PY{l+s}{r\PYZsq{}}\PY{l+s}{Temperature [\PYZdl{}\PYZca{}}\PY{l+s}{\PYZbs{}}\PY{l+s}{circ C\PYZdl{}] }\PY{l+s}{\PYZsq{}}\PY{p}{,}\PY{n}{fontsize} \PY{o}{=} \PY{n}{font\PYZus{}size}\PY{p}{)}
       
       \PY{n}{Cape\PYZus{}Weather}\PY{o}{.}\PY{n}{high}\PY{o}{.}\PY{n}{plot}\PY{p}{(}\PY{p}{)}
       \PY{n}{Cape\PYZus{}Weather}\PY{o}{.}\PY{n}{low}\PY{o}{.}\PY{n}{plot}\PY{p}{(}\PY{p}{)}
       \PY{n}{show}\PY{p}{(}\PY{p}{)}
\end{Verbatim}

    \begin{Verbatim}[commandchars=\\\{\}]
{\color{incolor}In [{\color{incolor}}]:} \PY{c}{\PYZsh{}CustomPlot()}
       \PY{n}{figsize}\PY{p}{(}\PY{l+m+mi}{22}\PY{p}{,} \PY{l+m+mi}{9}\PY{p}{)}
       \PY{n}{font\PYZus{}size} \PY{o}{=} \PY{l+m+mi}{24}
       
       \PY{n}{title}\PY{p}{(} \PY{l+s}{\PYZsq{}}\PY{l+s}{Mean solar radiation(horisontal plane)}\PY{l+s}{\PYZsq{}}\PY{p}{,} \PY{n}{fontsize}\PY{o}{=}\PY{n}{font\PYZus{}size}\PY{p}{)}
       \PY{n}{xlabel}\PY{p}{(}\PY{l+s}{\PYZsq{}}\PY{l+s}{Month Number}\PY{l+s}{\PYZsq{}}\PY{p}{,} \PY{n}{fontsize} \PY{o}{=} \PY{n}{font\PYZus{}size}\PY{p}{)}
       \PY{n}{ylabel}\PY{p}{(}\PY{l+s}{r\PYZsq{}}\PY{l+s}{\PYZdl{}MJ / day / m\PYZca{}2\PYZdl{}}\PY{l+s}{\PYZsq{}}\PY{p}{,}\PY{n}{fontsize} \PY{o}{=} \PY{n}{font\PYZus{}size}\PY{p}{)}
       
       \PY{n}{Cape\PYZus{}Weather}\PY{o}{.}\PY{n}{radiation}\PY{p}{[}\PY{l+m+mi}{0}\PY{p}{:}\PY{l+m+mi}{11}\PY{p}{]}\PY{o}{.}\PY{n}{plot}\PY{p}{(}\PY{p}{)}
       \PY{n}{show}\PY{p}{(}\PY{p}{)}
\end{Verbatim}

    \begin{Verbatim}[commandchars=\\\{\}]
{\color{incolor}In [{\color{incolor}}]:} \PY{c}{\PYZsh{} lets look at a proxy for heating degree and cooling degree days}
       
       \PY{n}{level} \PY{o}{=} \PY{l+m+mi}{25}
       
       \PY{k}{print} \PY{l+s}{\PYZsq{}}\PY{l+s}{\PYZgt{}25 =}\PY{l+s}{\PYZsq{}}\PY{p}{,} \PY{n}{Cape\PYZus{}Weather}\PY{p}{[} \PY{n}{Cape\PYZus{}Weather}\PY{p}{[}\PY{l+s}{\PYZsq{}}\PY{l+s}{high}\PY{l+s}{\PYZsq{}}\PY{p}{]} \PY{o}{\PYZgt{}} \PY{n}{level}  \PY{p}{]}\PY{o}{.}\PY{n}{count}\PY{p}{(}\PY{p}{)}\PY{p}{[}\PY{l+s}{\PYZsq{}}\PY{l+s}{high}\PY{l+s}{\PYZsq{}}\PY{p}{]}
       \PY{k}{print} \PY{l+s}{\PYZsq{}}\PY{l+s}{\PYZlt{}=25 =}\PY{l+s}{\PYZsq{}}\PY{p}{,} \PY{n}{Cape\PYZus{}Weather}\PY{p}{[} \PY{n}{Cape\PYZus{}Weather}\PY{p}{[}\PY{l+s}{\PYZsq{}}\PY{l+s}{high}\PY{l+s}{\PYZsq{}}\PY{p}{]} \PY{o}{\PYZlt{}}\PY{o}{=} \PY{n}{level} \PY{p}{]}\PY{o}{.}\PY{n}{count}\PY{p}{(}\PY{p}{)}\PY{p}{[}\PY{l+s}{\PYZsq{}}\PY{l+s}{high}\PY{l+s}{\PYZsq{}}\PY{p}{]}
\end{Verbatim}

    \begin{Verbatim}[commandchars=\\\{\}]
{\color{incolor}In [{\color{incolor}}]:} \PY{c}{\PYZsh{} Basic descriptive statistics}
       \PY{k}{print} \PY{n}{Cape\PYZus{}Weather}\PY{o}{.}\PY{n}{describe}\PY{p}{(}\PY{p}{)}
\end{Verbatim}

    \section{Run a SQL query on a dataframe!}

\begin{itemize}
\itemsep1pt\parskip0pt\parsep0pt
\item
  the pandasql module allows you to query the df as if it is a sqlite
  database
\end{itemize}

    \begin{Verbatim}[commandchars=\\\{\}]
{\color{incolor}In [{\color{incolor}}]:} \PY{k+kn}{from} \PY{n+nn}{pandasql} \PY{k+kn}{import} \PY{n}{sqldf}
       
       \PY{c}{\PYZsh{}helper function to extract df from the globals list}
       \PY{n}{pysqldf} \PY{o}{=} \PY{k}{lambda} \PY{n}{q}\PY{p}{:} \PY{n}{sqldf}\PY{p}{(}\PY{n}{q}\PY{p}{,} \PY{n+nb}{globals}\PY{p}{(}\PY{p}{)}\PY{p}{)}
       
       \PY{n}{temp\PYZus{}range} \PY{o}{=} \PY{n}{pysqldf}\PY{p}{(} \PY{l+s}{\PYZsq{}\PYZsq{}\PYZsq{}}\PY{l+s}{ }
       \PY{l+s}{                         select                   }
       \PY{l+s}{                             count(high) Count,     }
       \PY{l+s}{                             (high\PYZhy{}low) T\PYZus{}spread  }
       \PY{l+s}{                             }
       \PY{l+s}{                         from                         }
       \PY{l+s}{                            Cape\PYZus{}Weather  }
       \PY{l+s}{                            }
       \PY{l+s}{                         where                       }
       \PY{l+s}{                            high \PYZgt{} 25 and   }
       \PY{l+s}{                            low \PYZlt{} 10}\PY{l+s}{\PYZsq{}\PYZsq{}\PYZsq{}} \PY{p}{)}              
       \PY{n}{temp\PYZus{}range}
\end{Verbatim}

    \begin{Verbatim}[commandchars=\\\{\}]
{\color{incolor}In [{\color{incolor}}]:} \PY{c}{\PYZsh{}CustomPlot()}
       \PY{n}{figsize}\PY{p}{(}\PY{l+m+mi}{22}\PY{p}{,} \PY{l+m+mi}{9}\PY{p}{)}
       \PY{n}{font\PYZus{}size} \PY{o}{=} \PY{l+m+mi}{24}
       \PY{n}{title}\PY{p}{(}\PY{l+s}{\PYZsq{}}\PY{l+s}{Cape Town temperature distribution}\PY{l+s}{\PYZsq{}}\PY{p}{,} \PY{n}{fontsize}\PY{o}{=}\PY{n}{font\PYZus{}size}\PY{p}{)}
       \PY{n}{ylabel}\PY{p}{(}\PY{l+s}{\PYZsq{}}\PY{l+s}{Day count}\PY{l+s}{\PYZsq{}}\PY{p}{,}\PY{n}{fontsize} \PY{o}{=} \PY{n}{font\PYZus{}size}\PY{p}{)}
       \PY{n}{xlabel}\PY{p}{(}\PY{l+s}{r\PYZsq{}}\PY{l+s}{Temperature [\PYZdl{}\PYZca{}}\PY{l+s}{\PYZbs{}}\PY{l+s}{circ C \PYZdl{}] }\PY{l+s}{\PYZsq{}}\PY{p}{,}\PY{n}{fontsize} \PY{o}{=} \PY{n}{font\PYZus{}size}\PY{p}{)}
       \PY{n}{Cape\PYZus{}Weather}\PY{p}{[}\PY{l+s}{\PYZsq{}}\PY{l+s}{high}\PY{l+s}{\PYZsq{}}\PY{p}{]}\PY{o}{.}\PY{n}{hist}\PY{p}{(}\PY{n}{bins}\PY{o}{=}\PY{l+m+mi}{10}\PY{p}{)}
       \PY{n}{show}\PY{p}{(}\PY{p}{)}
\end{Verbatim}

    \subsection{Typesetting}

\subsubsection{LaTex}

\begin{itemize}
\itemsep1pt\parskip0pt\parsep0pt
\item
  LaTex is rendered using the mathjax.js JavaScript library
\end{itemize}

    \begin{Verbatim}[commandchars=\\\{\}]
{\color{incolor}In [{\color{incolor}}]:} \PY{k+kn}{from} \PY{n+nn}{IPython.display} \PY{k+kn}{import} \PY{n}{Math}
       
       \PY{n}{Math}\PY{p}{(}\PY{l+s}{r\PYZsq{}}\PY{l+s}{F(k) = }\PY{l+s}{\PYZbs{}}\PY{l+s}{int\PYZus{}\PYZob{}\PYZhy{}}\PY{l+s}{\PYZbs{}}\PY{l+s}{infty\PYZcb{}\PYZca{}\PYZob{}}\PY{l+s}{\PYZbs{}}\PY{l+s}{infty\PYZcb{} f(x) e\PYZca{}\PYZob{}2}\PY{l+s}{\PYZbs{}}\PY{l+s}{pi i k\PYZcb{} dx}\PY{l+s}{\PYZsq{}}\PY{p}{)}
\end{Verbatim}

    \begin{Verbatim}[commandchars=\\\{\}]
{\color{incolor}In [{\color{incolor}}]:} \PY{k+kn}{from} \PY{n+nn}{IPython.display} \PY{k+kn}{import} \PY{n}{Latex}
       \PY{n}{Latex}\PY{p}{(}\PY{l+s}{r\PYZdq{}\PYZdq{}\PYZdq{}}\PY{l+s}{\PYZbs{}}\PY{l+s}{begin\PYZob{}eqnarray\PYZcb{}}
       \PY{l+s}{\PYZbs{}}\PY{l+s}{nabla }\PY{l+s}{\PYZbs{}}\PY{l+s}{times }\PY{l+s}{\PYZbs{}}\PY{l+s}{vec\PYZob{}}\PY{l+s}{\PYZbs{}}\PY{l+s}{mathbf\PYZob{}B\PYZcb{}\PYZcb{} \PYZhy{}}\PY{l+s}{\PYZbs{}}\PY{l+s}{, }\PY{l+s}{\PYZbs{}}\PY{l+s}{frac1c}\PY{l+s}{\PYZbs{}}\PY{l+s}{, }\PY{l+s}{\PYZbs{}}\PY{l+s}{frac\PYZob{}}\PY{l+s}{\PYZbs{}}\PY{l+s}{partial}\PY{l+s}{\PYZbs{}}\PY{l+s}{vec\PYZob{}}\PY{l+s}{\PYZbs{}}\PY{l+s}{mathbf\PYZob{}E\PYZcb{}\PYZcb{}\PYZcb{}\PYZob{}}\PY{l+s}{\PYZbs{}}\PY{l+s}{partial t\PYZcb{} \PYZam{} = }\PY{l+s}{\PYZbs{}}\PY{l+s}{frac\PYZob{}4}\PY{l+s}{\PYZbs{}}\PY{l+s}{pi\PYZcb{}\PYZob{}c\PYZcb{}}\PY{l+s}{\PYZbs{}}\PY{l+s}{vec\PYZob{}}\PY{l+s}{\PYZbs{}}\PY{l+s}{mathbf\PYZob{}j\PYZcb{}\PYZcb{} }\PY{l+s}{\PYZbs{}}\PY{l+s}{\PYZbs{}}
       \PY{l+s}{\PYZbs{}}\PY{l+s}{nabla }\PY{l+s}{\PYZbs{}}\PY{l+s}{cdot }\PY{l+s}{\PYZbs{}}\PY{l+s}{vec\PYZob{}}\PY{l+s}{\PYZbs{}}\PY{l+s}{mathbf\PYZob{}E\PYZcb{}\PYZcb{} \PYZam{} = 4 }\PY{l+s}{\PYZbs{}}\PY{l+s}{pi }\PY{l+s}{\PYZbs{}}\PY{l+s}{rho }\PY{l+s}{\PYZbs{}}\PY{l+s}{\PYZbs{}}
       \PY{l+s}{\PYZbs{}}\PY{l+s}{nabla }\PY{l+s}{\PYZbs{}}\PY{l+s}{times }\PY{l+s}{\PYZbs{}}\PY{l+s}{vec\PYZob{}}\PY{l+s}{\PYZbs{}}\PY{l+s}{mathbf\PYZob{}E\PYZcb{}\PYZcb{}}\PY{l+s}{\PYZbs{}}\PY{l+s}{, +}\PY{l+s}{\PYZbs{}}\PY{l+s}{, }\PY{l+s}{\PYZbs{}}\PY{l+s}{frac1c}\PY{l+s}{\PYZbs{}}\PY{l+s}{, }\PY{l+s}{\PYZbs{}}\PY{l+s}{frac\PYZob{}}\PY{l+s}{\PYZbs{}}\PY{l+s}{partial}\PY{l+s}{\PYZbs{}}\PY{l+s}{vec\PYZob{}}\PY{l+s}{\PYZbs{}}\PY{l+s}{mathbf\PYZob{}B\PYZcb{}\PYZcb{}\PYZcb{}\PYZob{}}\PY{l+s}{\PYZbs{}}\PY{l+s}{partial t\PYZcb{} \PYZam{} = }\PY{l+s}{\PYZbs{}}\PY{l+s}{vec\PYZob{}}\PY{l+s}{\PYZbs{}}\PY{l+s}{mathbf\PYZob{}0\PYZcb{}\PYZcb{} }\PY{l+s}{\PYZbs{}}\PY{l+s}{\PYZbs{}}
       \PY{l+s}{\PYZbs{}}\PY{l+s}{nabla }\PY{l+s}{\PYZbs{}}\PY{l+s}{cdot }\PY{l+s}{\PYZbs{}}\PY{l+s}{vec\PYZob{}}\PY{l+s}{\PYZbs{}}\PY{l+s}{mathbf\PYZob{}B\PYZcb{}\PYZcb{} \PYZam{} = 0 }
       \PY{l+s}{\PYZbs{}}\PY{l+s}{end\PYZob{}eqnarray\PYZcb{}}\PY{l+s}{\PYZdq{}\PYZdq{}\PYZdq{}}\PY{p}{)}
\end{Verbatim}

    \subsection{Saving a Gist}

\begin{itemize}
\itemsep1pt\parskip0pt\parsep0pt
\item
  It is possible to save specific lines of code to a GitHub gist.
\item
  This is achieved with the \texttt{\%pastebin} magic as demonstrated
  below.
\item
  This makes sharing code easy!
\end{itemize}

    \begin{Verbatim}[commandchars=\\\{\}]
{\color{incolor}In [{\color{incolor}}]:} \PY{o}{\PYZpc{}}\PY{k}{pastebin} \PY{l+s}{\PYZdq{}}\PY{l+s}{cell one}\PY{l+s}{\PYZdq{}} \PY{l+m+mi}{0}\PY{o}{\PYZhy{}}\PY{l+m+mi}{10}
\end{Verbatim}

    \subsection{Connect to this kernel remotely}

\begin{itemize}
\item
  Using the \emph{\%connect}info\_ magic you can obtain the connection
  info to connect to this workbook from another IPython console or
  qtconsole using :
\item
  \emph{\emph{ipython qtconsole --existing}}
\item
  Using the port, signature and key you can also connect to a remote
  IPython kernel via SSH
\end{itemize}

    \begin{Verbatim}[commandchars=\\\{\}]
{\color{incolor}In [{\color{incolor}}]:} \PY{o}{\PYZpc{}}\PY{k}{connect\PYZus{}info}
       \PY{c}{\PYZsh{} lets connect to this kernel using the command \PYZdq{}ipython qtconsole \PYZhy{}\PYZhy{}existing\PYZdq{}}
\end{Verbatim}

    \section{Adding Interactivity}

\begin{itemize}
\itemsep1pt\parskip0pt\parsep0pt
\item
  IPython includes an architecture for interactive widgets
\item
  Ties together Python code running in the kernel and
  JavaScript/HTML/CSS in the browser
\item
  These widgets enable users to explore their code and data
  interactively
\end{itemize}

    \begin{Verbatim}[commandchars=\\\{\}]
{\color{incolor}In [{\color{incolor}}]:} \PY{k+kn}{from} \PY{n+nn}{IPython.html.widgets} \PY{k+kn}{import} \PY{n}{interact}
       \PY{k+kn}{import} \PY{n+nn}{matplotlib.pyplot} \PY{k+kn}{as} \PY{n+nn}{plt}
       \PY{k+kn}{import} \PY{n+nn}{networkx} \PY{k+kn}{as} \PY{n+nn}{nx}
       
       \PY{c}{\PYZsh{} wrap a few graph generation functions so they have the same signature}
       
       \PY{k}{def} \PY{n+nf}{random\PYZus{}lobster}\PY{p}{(}\PY{n}{n}\PY{p}{,} \PY{n}{m}\PY{p}{,} \PY{n}{k}\PY{p}{,} \PY{n}{p}\PY{p}{)}\PY{p}{:}
           \PY{k}{return} \PY{n}{nx}\PY{o}{.}\PY{n}{random\PYZus{}lobster}\PY{p}{(}\PY{n}{n}\PY{p}{,} \PY{n}{p}\PY{p}{,} \PY{n}{p} \PY{o}{/} \PY{n}{m}\PY{p}{)}
       
       \PY{k}{def} \PY{n+nf}{powerlaw\PYZus{}cluster}\PY{p}{(}\PY{n}{n}\PY{p}{,} \PY{n}{m}\PY{p}{,} \PY{n}{k}\PY{p}{,} \PY{n}{p}\PY{p}{)}\PY{p}{:}
           \PY{k}{return} \PY{n}{nx}\PY{o}{.}\PY{n}{powerlaw\PYZus{}cluster\PYZus{}graph}\PY{p}{(}\PY{n}{n}\PY{p}{,} \PY{n}{m}\PY{p}{,} \PY{n}{p}\PY{p}{)}
       
       \PY{k}{def} \PY{n+nf}{erdos\PYZus{}renyi}\PY{p}{(}\PY{n}{n}\PY{p}{,} \PY{n}{m}\PY{p}{,} \PY{n}{k}\PY{p}{,} \PY{n}{p}\PY{p}{)}\PY{p}{:}
           \PY{k}{return} \PY{n}{nx}\PY{o}{.}\PY{n}{erdos\PYZus{}renyi\PYZus{}graph}\PY{p}{(}\PY{n}{n}\PY{p}{,} \PY{n}{p}\PY{p}{)}
       
       \PY{k}{def} \PY{n+nf}{newman\PYZus{}watts\PYZus{}strogatz}\PY{p}{(}\PY{n}{n}\PY{p}{,} \PY{n}{m}\PY{p}{,} \PY{n}{k}\PY{p}{,} \PY{n}{p}\PY{p}{)}\PY{p}{:}
           \PY{k}{return} \PY{n}{nx}\PY{o}{.}\PY{n}{newman\PYZus{}watts\PYZus{}strogatz\PYZus{}graph}\PY{p}{(}\PY{n}{n}\PY{p}{,} \PY{n}{k}\PY{p}{,} \PY{n}{p}\PY{p}{)}
       
       \PY{k}{def} \PY{n+nf}{plot\PYZus{}random\PYZus{}graph}\PY{p}{(}\PY{n}{n}\PY{p}{,} \PY{n}{m}\PY{p}{,} \PY{n}{k}\PY{p}{,} \PY{n}{p}\PY{p}{,} \PY{n}{generator}\PY{p}{)}\PY{p}{:}
           \PY{n}{g} \PY{o}{=} \PY{n}{generator}\PY{p}{(}\PY{n}{n}\PY{p}{,} \PY{n}{m}\PY{p}{,} \PY{n}{k}\PY{p}{,} \PY{n}{p}\PY{p}{)}
           \PY{n}{nx}\PY{o}{.}\PY{n}{draw}\PY{p}{(}\PY{n}{g}\PY{p}{)}
           \PY{n}{plt}\PY{o}{.}\PY{n}{show}\PY{p}{(}\PY{p}{)}
\end{Verbatim}

    \begin{Verbatim}[commandchars=\\\{\}]
{\color{incolor}In [{\color{incolor}}]:} \PY{c}{\PYZsh{}static}
       \PY{n}{plot\PYZus{}random\PYZus{}graph}\PY{p}{(} \PY{l+m+mi}{18}\PY{p}{,} \PY{l+m+mi}{5}\PY{p}{,} \PY{l+m+mi}{5}\PY{p}{,} \PY{l+m+mf}{0.449}\PY{p}{,} \PY{n}{newman\PYZus{}watts\PYZus{}strogatz}\PY{p}{)}
\end{Verbatim}

    \begin{Verbatim}[commandchars=\\\{\}]
{\color{incolor}In [{\color{incolor}}]:} \PY{n}{interact}\PY{p}{(}
               \PY{n}{plot\PYZus{}random\PYZus{}graph}\PY{p}{,} \PY{n}{n}\PY{o}{=}\PY{p}{(}\PY{l+m+mi}{2}\PY{p}{,}\PY{l+m+mi}{30}\PY{p}{)}\PY{p}{,} \PY{n}{m}\PY{o}{=}\PY{p}{(}\PY{l+m+mi}{1}\PY{p}{,}\PY{l+m+mi}{10}\PY{p}{)}\PY{p}{,} \PY{n}{k}\PY{o}{=}\PY{p}{(}\PY{l+m+mi}{1}\PY{p}{,}\PY{l+m+mi}{10}\PY{p}{)}\PY{p}{,} \PY{n}{p}\PY{o}{=}\PY{p}{(}\PY{l+m+mf}{0.0}\PY{p}{,} \PY{l+m+mf}{1.0}\PY{p}{,} \PY{l+m+mf}{0.001}\PY{p}{)}\PY{p}{,}
               \PY{n}{generator}\PY{o}{=}\PY{p}{\PYZob{}}\PY{l+s}{\PYZsq{}}\PY{l+s}{lobster}\PY{l+s}{\PYZsq{}}\PY{p}{:} \PY{n}{random\PYZus{}lobster}\PY{p}{,}
                          \PY{l+s}{\PYZsq{}}\PY{l+s}{power law}\PY{l+s}{\PYZsq{}}\PY{p}{:} \PY{n}{powerlaw\PYZus{}cluster}\PY{p}{,}
                          \PY{l+s}{\PYZsq{}}\PY{l+s}{Newman\PYZhy{}Watts\PYZhy{}Strogatz}\PY{l+s}{\PYZsq{}}\PY{p}{:} \PY{n}{newman\PYZus{}watts\PYZus{}strogatz}\PY{p}{,}
                          \PY{l+s}{u\PYZsq{}}\PY{l+s}{Erdős\PYZhy{}Rényi}\PY{l+s}{\PYZsq{}}\PY{p}{:} \PY{n}{erdos\PYZus{}renyi}\PY{p}{,}
                          \PY{p}{\PYZcb{}}\PY{p}{)}\PY{p}{;}
\end{Verbatim}

    \begin{Verbatim}[commandchars=\\\{\}]
{\color{incolor}In [{\color{incolor}}]:} \PY{k+kn}{from} \PY{n+nn}{IPython.html.widgets} \PY{k+kn}{import} \PY{n}{interact}
       \PY{k+kn}{import} \PY{n+nn}{matplotlib.pyplot} \PY{k+kn}{as} \PY{n+nn}{plt}
       \PY{k+kn}{import} \PY{n+nn}{numpy} \PY{k+kn}{as} \PY{n+nn}{np}
       
       \PY{o}{\PYZpc{}}\PY{k}{matplotlib} \PY{n}{inline}
       
       \PY{n}{np}\PY{o}{.}\PY{n}{random}\PY{o}{.}\PY{n}{seed}\PY{p}{(}\PY{l+m+mi}{0}\PY{p}{)}
       \PY{n}{x}\PY{p}{,} \PY{n}{y} \PY{o}{=} \PY{n}{np}\PY{o}{.}\PY{n}{random}\PY{o}{.}\PY{n}{normal}\PY{p}{(}\PY{n}{size}\PY{o}{=}\PY{p}{(}\PY{l+m+mi}{2}\PY{p}{,} \PY{l+m+mi}{100}\PY{p}{)}\PY{p}{)}
       \PY{n}{s}\PY{p}{,} \PY{n}{c} \PY{o}{=} \PY{n}{np}\PY{o}{.}\PY{n}{random}\PY{o}{.}\PY{n}{random}\PY{p}{(}\PY{n}{size}\PY{o}{=}\PY{p}{(}\PY{l+m+mi}{2}\PY{p}{,} \PY{l+m+mi}{100}\PY{p}{)}\PY{p}{)}
           
       \PY{k}{def} \PY{n+nf}{draw\PYZus{}scatter}\PY{p}{(}\PY{n}{size}\PY{o}{=}\PY{l+m+mi}{100}\PY{p}{,} \PY{n}{cmap}\PY{o}{=}\PY{l+s}{\PYZsq{}}\PY{l+s}{jet}\PY{l+s}{\PYZsq{}}\PY{p}{,} \PY{n}{alpha}\PY{o}{=}\PY{l+m+mf}{1.0}\PY{p}{)}\PY{p}{:}
           \PY{n}{fig}\PY{p}{,} \PY{n}{ax} \PY{o}{=} \PY{n}{plt}\PY{o}{.}\PY{n}{subplots}\PY{p}{(}\PY{n}{figsize}\PY{o}{=}\PY{p}{(}\PY{l+m+mi}{20}\PY{p}{,} \PY{l+m+mi}{8}\PY{p}{)}\PY{p}{)}
           \PY{n}{points} \PY{o}{=} \PY{n}{ax}\PY{o}{.}\PY{n}{scatter}\PY{p}{(}\PY{n}{x}\PY{p}{,} \PY{n}{y}\PY{p}{,} \PY{n}{s}\PY{o}{=}\PY{n}{size}\PY{o}{*}\PY{n}{s}\PY{p}{,} \PY{n}{c}\PY{o}{=}\PY{n}{c}\PY{p}{,} \PY{n}{alpha}\PY{o}{=}\PY{n}{alpha}\PY{p}{,} \PY{n}{cmap}\PY{o}{=}\PY{n}{cmap}\PY{p}{)}
           \PY{n}{fig}\PY{o}{.}\PY{n}{colorbar}\PY{p}{(}\PY{n}{points}\PY{p}{,} \PY{n}{ax}\PY{o}{=}\PY{n}{ax}\PY{p}{)}
           \PY{k}{return} \PY{n}{fig}
       
       \PY{n}{colormaps} \PY{o}{=} \PY{n+nb}{sorted}\PY{p}{(}\PY{n}{m} \PY{k}{for} \PY{n}{m} \PY{o+ow}{in} \PY{n}{plt}\PY{o}{.}\PY{n}{cm}\PY{o}{.}\PY{n}{datad} \PY{k}{if} \PY{o+ow}{not} \PY{n}{m}\PY{o}{.}\PY{n}{endswith}\PY{p}{(}\PY{l+s}{\PYZdq{}}\PY{l+s}{\PYZus{}r}\PY{l+s}{\PYZdq{}}\PY{p}{)}\PY{p}{)}
\end{Verbatim}

    \begin{Verbatim}[commandchars=\\\{\}]
{\color{incolor}In [{\color{incolor}}]:} \PY{n}{interact}\PY{p}{(}
                   \PY{n}{draw\PYZus{}scatter}\PY{p}{,} \PY{n}{size}\PY{o}{=}\PY{p}{[}\PY{l+m+mi}{0}\PY{p}{,} \PY{l+m+mi}{2000}\PY{p}{]}\PY{p}{,} \PY{n}{alpha}\PY{o}{=}\PY{p}{[}\PY{l+m+mf}{0.0}\PY{p}{,} \PY{l+m+mf}{1.0}\PY{p}{]}\PY{p}{,} \PY{n}{cmap}\PY{o}{=}\PY{n}{colormaps}
               \PY{p}{)}\PY{p}{;}
\end{Verbatim}

    \section{Exploring Beat Frequencies using the Audio Object}

\begin{itemize}
\itemsep1pt\parskip0pt\parsep0pt
\item
  This example uses the Audio object and Matplotlib to explore the
  phenomenon of beat frequencies.
\end{itemize}

    \begin{Verbatim}[commandchars=\\\{\}]
{\color{incolor}In [{\color{incolor}}]:} \PY{o}{\PYZpc{}}\PY{k}{matplotlib} \PY{n}{inline}
       \PY{k+kn}{import} \PY{n+nn}{matplotlib.pyplot} \PY{k+kn}{as} \PY{n+nn}{plt}
       \PY{k+kn}{import} \PY{n+nn}{numpy} \PY{k+kn}{as} \PY{n+nn}{np}
       
       \PY{k+kn}{from} \PY{n+nn}{IPython.html.widgets} \PY{k+kn}{import} \PY{n}{interactive}
       \PY{k+kn}{from} \PY{n+nn}{IPython.display} \PY{k+kn}{import} \PY{n}{Audio}\PY{p}{,} \PY{n}{display}
       \PY{k+kn}{import} \PY{n+nn}{numpy} \PY{k+kn}{as} \PY{n+nn}{np}
       
       \PY{k}{def} \PY{n+nf}{beat\PYZus{}freq}\PY{p}{(}\PY{n}{f1}\PY{o}{=}\PY{l+m+mf}{220.0}\PY{p}{,} \PY{n}{f2}\PY{o}{=}\PY{l+m+mf}{224.0}\PY{p}{)}\PY{p}{:}
           \PY{n}{max\PYZus{}time} \PY{o}{=} \PY{l+m+mi}{3}
           \PY{n}{rate} \PY{o}{=} \PY{l+m+mi}{8000}
           \PY{n}{times} \PY{o}{=} \PY{n}{np}\PY{o}{.}\PY{n}{linspace}\PY{p}{(}\PY{l+m+mi}{0}\PY{p}{,}\PY{n}{max\PYZus{}time}\PY{p}{,}\PY{n}{rate}\PY{o}{*}\PY{n}{max\PYZus{}time}\PY{p}{)}
           \PY{n}{signal} \PY{o}{=} \PY{n}{np}\PY{o}{.}\PY{n}{sin}\PY{p}{(}\PY{l+m+mi}{2}\PY{o}{*}\PY{n}{np}\PY{o}{.}\PY{n}{pi}\PY{o}{*}\PY{n}{f1}\PY{o}{*}\PY{n}{times}\PY{p}{)} \PY{o}{+} \PY{n}{np}\PY{o}{.}\PY{n}{sin}\PY{p}{(}\PY{l+m+mi}{2}\PY{o}{*}\PY{n}{np}\PY{o}{.}\PY{n}{pi}\PY{o}{*}\PY{n}{f2}\PY{o}{*}\PY{n}{times}\PY{p}{)}
           \PY{k}{print}\PY{p}{(}\PY{n}{f1}\PY{p}{,} \PY{n}{f2}\PY{p}{,} \PY{n+nb}{abs}\PY{p}{(}\PY{n}{f1}\PY{o}{\PYZhy{}}\PY{n}{f2}\PY{p}{)}\PY{p}{)}
           \PY{n}{display}\PY{p}{(}\PY{n}{Audio}\PY{p}{(}\PY{n}{data}\PY{o}{=}\PY{n}{signal}\PY{p}{,} \PY{n}{rate}\PY{o}{=}\PY{n}{rate}\PY{p}{)}\PY{p}{)}
           \PY{k}{return} \PY{n}{signal}
\end{Verbatim}

    \begin{Verbatim}[commandchars=\\\{\}]
{\color{incolor}In [{\color{incolor}}]:} \PY{n}{v} \PY{o}{=} \PY{n}{interactive}\PY{p}{(}\PY{n}{beat\PYZus{}freq}\PY{p}{,} \PY{n}{f1}\PY{o}{=}\PY{p}{(}\PY{l+m+mf}{200.0}\PY{p}{,}\PY{l+m+mf}{300.0}\PY{p}{)}\PY{p}{,} \PY{n}{f2}\PY{o}{=}\PY{p}{(}\PY{l+m+mf}{200.0}\PY{p}{,}\PY{l+m+mf}{300.0}\PY{p}{)}\PY{p}{)}
       \PY{n}{display}\PY{p}{(}\PY{n}{v}\PY{p}{)}
\end{Verbatim}

    \subsection{Lorenz System of Differential Equations}

\begin{itemize}
\itemsep1pt\parskip0pt\parsep0pt
\item
  Also example to link to other notebooks
\item
  Click \href{static/notebook/LDE.ipynb}{Lorenz Differential Equations}
\end{itemize}

    \section{Machine Learning with Sci-Kit Learn}

\begin{figure}[htbp]
\centering
\includegraphics{static/img/scikit.jpg}
\end{figure}

    \begin{Verbatim}[commandchars=\\\{\}]
{\color{incolor}In [{\color{incolor}}]:} \PY{n}{solve} \PY{o}{=} \PY{n+nb+bp}{True}
       
       \PY{k+kn}{import} \PY{n+nn}{pylab} \PY{k+kn}{as} \PY{n+nn}{pl}
       \PY{k+kn}{import} \PY{n+nn}{numpy} \PY{k+kn}{as} \PY{n+nn}{np}
       
       \PY{k+kn}{from} \PY{n+nn}{sklearn.ensemble} \PY{k+kn}{import} \PY{n}{AdaBoostClassifier}
       \PY{k+kn}{from} \PY{n+nn}{sklearn.tree} \PY{k+kn}{import} \PY{n}{DecisionTreeClassifier}
       \PY{k+kn}{from} \PY{n+nn}{sklearn.datasets} \PY{k+kn}{import} \PY{n}{make\PYZus{}gaussian\PYZus{}quantiles}
       
       
       \PY{c}{\PYZsh{} Construct dataset}
       \PY{n}{X1}\PY{p}{,} \PY{n}{y1} \PY{o}{=} \PY{n}{make\PYZus{}gaussian\PYZus{}quantiles}\PY{p}{(}\PY{n}{cov}\PY{o}{=}\PY{l+m+mf}{2.}\PY{p}{,}
                                        \PY{n}{n\PYZus{}samples}\PY{o}{=}\PY{l+m+mi}{200}\PY{p}{,} \PY{n}{n\PYZus{}features}\PY{o}{=}\PY{l+m+mi}{2}\PY{p}{,}
                                        \PY{n}{n\PYZus{}classes}\PY{o}{=}\PY{l+m+mi}{2}\PY{p}{,} \PY{n}{random\PYZus{}state}\PY{o}{=}\PY{l+m+mi}{1}\PY{p}{)}
       \PY{n}{X2}\PY{p}{,} \PY{n}{y2} \PY{o}{=} \PY{n}{make\PYZus{}gaussian\PYZus{}quantiles}\PY{p}{(}\PY{n}{mean}\PY{o}{=}\PY{p}{(}\PY{l+m+mi}{3}\PY{p}{,} \PY{l+m+mi}{3}\PY{p}{)}\PY{p}{,} \PY{n}{cov}\PY{o}{=}\PY{l+m+mf}{1.5}\PY{p}{,}
                                        \PY{n}{n\PYZus{}samples}\PY{o}{=}\PY{l+m+mi}{300}\PY{p}{,} \PY{n}{n\PYZus{}features}\PY{o}{=}\PY{l+m+mi}{2}\PY{p}{,}
                                        \PY{n}{n\PYZus{}classes}\PY{o}{=}\PY{l+m+mi}{2}\PY{p}{,} \PY{n}{random\PYZus{}state}\PY{o}{=}\PY{l+m+mi}{1}\PY{p}{)}
       \PY{c}{\PYZsh{}Training Samples}
       \PY{n}{X} \PY{o}{=} \PY{n}{np}\PY{o}{.}\PY{n}{concatenate}\PY{p}{(}\PY{p}{(}\PY{n}{X1}\PY{p}{,} \PY{n}{X2}\PY{p}{)}\PY{p}{)}
       
       \PY{c}{\PYZsh{}Training Target}
       \PY{n}{y} \PY{o}{=} \PY{n}{np}\PY{o}{.}\PY{n}{concatenate}\PY{p}{(}\PY{p}{(}\PY{n}{y1}\PY{p}{,} \PY{o}{\PYZhy{}} \PY{n}{y2} \PY{o}{+} \PY{l+m+mi}{1}\PY{p}{)}\PY{p}{)}
       
       \PY{c}{\PYZsh{} Create and fit an AdaBoosted decision tree}
       \PY{n}{bdt} \PY{o}{=} \PY{n}{AdaBoostClassifier}\PY{p}{(}
                                 \PY{n}{DecisionTreeClassifier}\PY{p}{(} \PY{n}{max\PYZus{}depth}\PY{o}{=}\PY{l+m+mi}{1}\PY{p}{)}\PY{p}{,}
                                                         \PY{n}{algorithm}\PY{o}{=}\PY{l+s}{\PYZdq{}}\PY{l+s}{SAMME}\PY{l+s}{\PYZdq{}}\PY{p}{,}
                                                         \PY{n}{n\PYZus{}estimators}\PY{o}{=}\PY{l+m+mi}{200}
                               \PY{p}{)}
       
       \PY{n}{bdt}\PY{o}{.}\PY{n}{fit}\PY{p}{(}\PY{n}{X}\PY{p}{,} \PY{n}{y}\PY{p}{)}
       
       \PY{n}{plot\PYZus{}colors} \PY{o}{=} \PY{l+s}{\PYZdq{}}\PY{l+s}{br}\PY{l+s}{\PYZdq{}}
       \PY{n}{plot\PYZus{}step} \PY{o}{=} \PY{l+m+mf}{0.05}
       \PY{n}{class\PYZus{}names} \PY{o}{=} \PY{l+s}{\PYZdq{}}\PY{l+s}{AB}\PY{l+s}{\PYZdq{}}
       
       \PY{n}{pl}\PY{o}{.}\PY{n}{figure}\PY{p}{(}\PY{n}{figsize}\PY{o}{=}\PY{p}{(}\PY{l+m+mi}{20}\PY{p}{,} \PY{l+m+mi}{10}\PY{p}{)}\PY{p}{)}
       
       \PY{n}{x\PYZus{}min}\PY{p}{,} \PY{n}{x\PYZus{}max} \PY{o}{=} \PY{n}{X}\PY{p}{[}\PY{p}{:}\PY{p}{,} \PY{l+m+mi}{0}\PY{p}{]}\PY{o}{.}\PY{n}{min}\PY{p}{(}\PY{p}{)} \PY{o}{\PYZhy{}} \PY{l+m+mi}{1}\PY{p}{,} \PY{n}{X}\PY{p}{[}\PY{p}{:}\PY{p}{,} \PY{l+m+mi}{0}\PY{p}{]}\PY{o}{.}\PY{n}{max}\PY{p}{(}\PY{p}{)} \PY{o}{+} \PY{l+m+mi}{1}
       \PY{n}{y\PYZus{}min}\PY{p}{,} \PY{n}{y\PYZus{}max} \PY{o}{=} \PY{n}{X}\PY{p}{[}\PY{p}{:}\PY{p}{,} \PY{l+m+mi}{1}\PY{p}{]}\PY{o}{.}\PY{n}{min}\PY{p}{(}\PY{p}{)} \PY{o}{\PYZhy{}} \PY{l+m+mi}{1}\PY{p}{,} \PY{n}{X}\PY{p}{[}\PY{p}{:}\PY{p}{,} \PY{l+m+mi}{1}\PY{p}{]}\PY{o}{.}\PY{n}{max}\PY{p}{(}\PY{p}{)} \PY{o}{+} \PY{l+m+mi}{1}
       
       \PY{k}{if} \PY{n}{solve}\PY{p}{:}
           \PY{c}{\PYZsh{} Plot the decision boundaries}
           \PY{n}{pl}\PY{o}{.}\PY{n}{subplot}\PY{p}{(}\PY{l+m+mi}{121}\PY{p}{)}
           
           \PY{n}{xx}\PY{p}{,} \PY{n}{yy} \PY{o}{=} \PY{n}{np}\PY{o}{.}\PY{n}{meshgrid}\PY{p}{(}\PY{n}{np}\PY{o}{.}\PY{n}{arange}\PY{p}{(}\PY{n}{x\PYZus{}min}\PY{p}{,} \PY{n}{x\PYZus{}max}\PY{p}{,} \PY{n}{plot\PYZus{}step}\PY{p}{)}\PY{p}{,}
                                \PY{n}{np}\PY{o}{.}\PY{n}{arange}\PY{p}{(}\PY{n}{y\PYZus{}min}\PY{p}{,} \PY{n}{y\PYZus{}max}\PY{p}{,} \PY{n}{plot\PYZus{}step}\PY{p}{)}\PY{p}{)}
       
           \PY{n}{Z} \PY{o}{=} \PY{n}{bdt}\PY{o}{.}\PY{n}{predict}\PY{p}{(}\PY{n}{np}\PY{o}{.}\PY{n}{c\PYZus{}}\PY{p}{[}\PY{n}{xx}\PY{o}{.}\PY{n}{ravel}\PY{p}{(}\PY{p}{)}\PY{p}{,} \PY{n}{yy}\PY{o}{.}\PY{n}{ravel}\PY{p}{(}\PY{p}{)}\PY{p}{]}\PY{p}{)}
       
           \PY{n}{Z} \PY{o}{=} \PY{n}{Z}\PY{o}{.}\PY{n}{reshape}\PY{p}{(}\PY{n}{xx}\PY{o}{.}\PY{n}{shape}\PY{p}{)}
           \PY{n}{cs} \PY{o}{=} \PY{n}{pl}\PY{o}{.}\PY{n}{contourf}\PY{p}{(}\PY{n}{xx}\PY{p}{,} \PY{n}{yy}\PY{p}{,} \PY{n}{Z}\PY{p}{,} \PY{n}{cmap}\PY{o}{=}\PY{n}{pl}\PY{o}{.}\PY{n}{cm}\PY{o}{.}\PY{n}{Paired}\PY{p}{)}
           \PY{n}{pl}\PY{o}{.}\PY{n}{axis}\PY{p}{(}\PY{l+s}{\PYZdq{}}\PY{l+s}{tight}\PY{l+s}{\PYZdq{}}\PY{p}{)}
       
       
       \PY{c}{\PYZsh{} Plot the training points}
       \PY{k}{for} \PY{n}{i}\PY{p}{,} \PY{n}{n}\PY{p}{,} \PY{n}{c} \PY{o+ow}{in} \PY{n+nb}{zip}\PY{p}{(}\PY{n+nb}{range}\PY{p}{(}\PY{l+m+mi}{2}\PY{p}{)}\PY{p}{,} \PY{n}{class\PYZus{}names}\PY{p}{,} \PY{n}{plot\PYZus{}colors}\PY{p}{)}\PY{p}{:}
           \PY{n}{idx} \PY{o}{=} \PY{n}{np}\PY{o}{.}\PY{n}{where}\PY{p}{(}\PY{n}{y} \PY{o}{==} \PY{n}{i}\PY{p}{)}
           \PY{n}{pl}\PY{o}{.}\PY{n}{scatter}\PY{p}{(}\PY{n}{X}\PY{p}{[}\PY{n}{idx}\PY{p}{,} \PY{l+m+mi}{0}\PY{p}{]}\PY{p}{,} \PY{n}{X}\PY{p}{[}\PY{n}{idx}\PY{p}{,} \PY{l+m+mi}{1}\PY{p}{]}\PY{p}{,}
                      \PY{n}{c}\PY{o}{=}\PY{n}{c}\PY{p}{,} \PY{n}{cmap}\PY{o}{=}\PY{n}{pl}\PY{o}{.}\PY{n}{cm}\PY{o}{.}\PY{n}{Paired}\PY{p}{,}
                      \PY{n}{label}\PY{o}{=}\PY{l+s}{\PYZdq{}}\PY{l+s}{Class }\PY{l+s+si}{\PYZpc{}s}\PY{l+s}{\PYZdq{}} \PY{o}{\PYZpc{}} \PY{n}{n}\PY{p}{,} \PY{n}{s}\PY{o}{=}\PY{l+m+mi}{100}\PY{p}{)}
       \PY{n}{pl}\PY{o}{.}\PY{n}{xlim}\PY{p}{(}\PY{n}{x\PYZus{}min}\PY{p}{,} \PY{n}{x\PYZus{}max}\PY{p}{)}
       \PY{n}{pl}\PY{o}{.}\PY{n}{ylim}\PY{p}{(}\PY{n}{y\PYZus{}min}\PY{p}{,} \PY{n}{y\PYZus{}max}\PY{p}{)}
       \PY{n}{pl}\PY{o}{.}\PY{n}{legend}\PY{p}{(}\PY{n}{loc}\PY{o}{=}\PY{l+s}{\PYZsq{}}\PY{l+s}{upper right}\PY{l+s}{\PYZsq{}}\PY{p}{)}
       \PY{n}{pl}\PY{o}{.}\PY{n}{xlabel}\PY{p}{(}\PY{l+s}{\PYZdq{}}\PY{l+s}{Decision Boundary}\PY{l+s}{\PYZdq{}}\PY{p}{,}\PY{n}{size} \PY{o}{=} \PY{l+m+mi}{20}\PY{p}{)}
\end{Verbatim}

    \subsection{Publishing your Work}

\begin{itemize}
\item
  Ability to convert an .ipynb notebook document file into various
  static formats.
\item
  Currently, nbconvert is provided as a command line tool, run as a
  script using IPython.
\end{itemize}

This page is converted and published to the following formats using this
tool:

\begin{itemize}
\item
  HTML
\item
  PDF (the PDF is created using wkhtml2pdf that takes the html file as
  an input)
\item
  LATEX
\item
  Reveal.js slideshow
\end{itemize}

    \begin{Verbatim}[commandchars=\\\{\}]
{\color{incolor}In [{\color{incolor}}]:} \PY{o}{!}ipython nbconvert pyconza\PYZus{}ipython.ipynb \PYZhy{}\PYZhy{}to slides \PYZhy{}\PYZhy{}post serve
\end{Verbatim}

    \begin{Verbatim}[commandchars=\\\{\}]
{\color{incolor}In [{\color{incolor}}]:} \PY{o}{!}ipython nbconvert pyconza\PYZus{}ipython.ipynb \PYZhy{}\PYZhy{}to html \PYZhy{}\PYZhy{}post serve
\end{Verbatim}

    \section{Writing Books!}

\begin{itemize}
\itemsep1pt\parskip0pt\parsep0pt
\item
  Need internet connection but check out
\item
  \href{http://camdavidsonpilon.github.io/Probabilistic-Programming-and-Bayesian-Methods-for-Hackers/}{Probabilistic-Programming-and-Bayesian-Methods-for-Hackers}
\end{itemize}

    \section{THANK YOU FOR YOUR TIME}

\begin{itemize}
\item
  Find all of this - http://bit.ly/pyconza-notebook
\item
  Join the Gauteng Python Users Group - GPUG
\item
  ask @tooblippe, @wasbeer

  \begin{itemize}
  \itemsep1pt\parskip0pt\parsep0pt
  \item
    meetup.com http://www.meetup.com/Gauteng-Python-Users-Group/
  \item
    Google groups \#gpugsa
  \item
    website - http://gautengpug.github.io/
  \item
    github - http://github.com/gautengpug
  \end{itemize}
\item
  Python, DataAnalysis, Python for kids - check out
  http://www.insightstack.co.za
\end{itemize}

    \begin{Verbatim}[commandchars=\\\{\}]
{\color{incolor}In [{\color{incolor}}]:} 
\end{Verbatim}

    \begin{Verbatim}[commandchars=\\\{\}]
{\color{incolor}In [{\color{incolor}}]:} 
\end{Verbatim}


    % Add a bibliography block to the postdoc
    
    
    
    \end{document}
